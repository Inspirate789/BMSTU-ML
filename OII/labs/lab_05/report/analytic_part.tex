\chapter{Аналитическая часть}

\section{Типология порядка слов}

Типология порядка слов в предложении — один из методов типологической классификации языков, учитывающий базовый порядок слов в предложении: подлежащего (subject), сказуемого (verb) и дополнения (object). Современное состояние типологии базового порядка слов представлено во Всемирном атласе языковых структур.~\cite{map}

Начало современному изучению типологии базового порядка составляющих в языках мира было положено во второй половине XX века американским лингвистом Джозефом Гринбергом. Гринберг выделил шесть базовых порядков составляющих в предложении — SOV, SVO, VSO, VOS, OVS, OSV — и установил некоторые импликативные отношения между этим и другими порядками. Так, в соответствии с описанной классификацией, русский язык имеет базовый порядок слов SVO. В современной лингвистике базовый порядок слов в предложении не считается достаточным для типологической классификации порядка слов в языках мира и определяющим все частные порядки.

\section{Цепи Маркова и n-граммы}

Цепь Маркова — последовательность случайных событий с конечным или счётным числом исходов, где вероятность наступления каждого события зависит только от состояния, достигнутого в предыдущем событии. Характеризуется тем свойством, что её будущее состояние зависит только от текущего состояния и не зависит от предыдущих.

N-граммы – это статистические модели, которые предсказывают следующее слово после n-1 слов на основе частоты их сочетания в обучающей выборке.

С помощью цепей Маркова можно рассчитать вероятность появления следующего элемента последовательности (например, n-го слова после n-1-граммы), не учитывая слишком широкий контекст. 

\section*{Вывод}

В данном разделе была описана типология порядка слов в предложениях на естественных языках, а также были рассмотрены понятия Марковских цепей и n-грамм.

\clearpage
