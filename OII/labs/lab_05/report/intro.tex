\ssr{ВВЕДЕНИЕ}

Генерация текстов на естественном языке является актуальной задачей в области искусственного интеллекта. В современных приложениях ИИ, таких как чат-боты, 
системы рекомендации и переводчики, генерация текстов часто используется для улучшения пользовательского опыта взаимодействия с разрабатываемым ПО.

Одним из методов генерации текстов является использование цепей Маркова. Цепи Маркова — это математическое представление вероятностной  модели последовательности случайных процессов, которые могут быть применены к генерации текста на основе его контекста и соседних слов.

Целью данной лабораторной работы является изучение потенциала использования цепей Маркова для генерации текстов на естественном языке. 

Задачи данной лабораторной работы:
\begin{enumerate}[label*=\arabic*.]
	\item Сформировать n-граммы для генерации с помощью цепи Маркова на основе обучающего текста.
	\item Сформировать тексты с различными затравочными начальными словами.
	\item Дать экспертную оценку <<человечности>> сформированных текстов.
	\item Провести исследование возможности генерации текста при наличии обучающей выборки, состоящей только из предложений <<кошка съела мышку>> и <<мышку съела кошка>>, оценить опасность работы с языками с нестрогим порядком слов (не обязательно SVO).
	\item Осуществить генерацию текстов с помощью инструмента из представленной методички. Оценить получаемый порядок слов в генерируемых предложениях для разной версии qwen2.5.
\end{enumerate}

\clearpage
