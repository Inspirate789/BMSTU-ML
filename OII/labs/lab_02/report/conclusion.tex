\ssr{ЗАКЛЮЧЕНИЕ}

В рамках лабораторной работы была создана система для следования в одномерном пространстве, при отсутствии данных о скорости <<лидера>>, но с известным расстоянием до него. Все поставленные задачи были выполнены.

\begin{enumerate}[label*=\arabic*)]
	\item Описаны общие этапы нечёткого логического вывода;
	\item Предложены функции принадлежности термам числовых значений признаков, описываемых используемыми лингвистическими переменными;
	\item Предложены правила для нечёткого логического вывода;
	\item Описан алгоритм Ларсена для нечёткого логического вывода;
	\item Описаны алгоритмы дефаззификации;
	\item Реализована система следования в одномерном пространстве с использованием нечётких переменных и правил для определения необходимой скорости автопилота;
	\item Измерена среднеквадратичная ошибка время вычисления скорости автопилота.
\end{enumerate}

Минимальное значение среднеквадратичного отклонения расстояния между лидером и автопилотом составило 0.1 м, максимальное --- 35.1 м. Для повышения точности реализованной нечёткой модели необходимо либо переработать базу правил, дополнив её новыми правилами или изменив существующие, либо увеличить число нечётких переменных.
