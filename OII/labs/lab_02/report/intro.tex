\ssr{ВВЕДЕНИЕ}

В последние два десятилетия резко возрос интерес к различным аспектам
проблемы интеллектуального управления. Одно из основных направлений, связанных с решением этой проблемы, состоит в использовании аппарата нечётких систем: нечётких множеств, нечёткой логики, нечёткого
моделирования и т. п. Применение этого аппарата приводит к построению нечётких систем управления различных классов, позволяющих решать задачи управления в ситуациях, когда традиционные методы неэффективны или даже вообще неприменимы из-за отсутствия достаточно
точного знания об объекте управления.~\cite{book}

Целью данной лабораторной работы является создание системы для следования в одномерном пространстве, при отсутствии данных о скорости <<лидера>>, но с известным расстоянием до него.

Задачи данной лабораторной работы:
\begin{enumerate}[label*=\arabic*)]
	\item описать общие этапы нечёткого логического вывода;
	\item предложить функции принадлежности термам числовых значений признаков, описываемых используемыми лингвистическими переменными;
	\item предложить правила для нечёткого логического вывода;
	\item описать алгоритм Ларсена для нечёткого логического вывода;
	\item описать алгоритмы дефаззификации;
	\item реализовать систему следования в одномерном пространстве с использованием нечётких переменных и правил для определения необходимой скорости автопилота;
	\item измерить среднеквадратичную ошибку и время вычисления скорости автопилота.
\end{enumerate}

\clearpage
