\ssr{ЗАКЛЮЧЕНИЕ}

В рамках лабораторной работы была составлена программа, которая, с использованием алгоритмов оптимизации (генетического и эволюционного на основе муравьиной кучи), аппроксимирует функции $f(x) = sin(x)*(sin(x)+cos(x)), x\in [-2\pi; +2\pi], g(x) = x^3 - x +3$.. Все поставленные задачи были выполнены.

\begin{enumerate}[label*=\arabic*)]
	\item описаны общие этапы функционирования системы;
	\item описан предлагаемый генетический алгоритм;
	\item описан алгоритм муравьиной кучи;
	\item реализован программу для аппроксимации заданных функций;
	\item измерены среднеквадратичную ошибку и время выполнения алгоритмов;
	\item проведено сравнение реализованных алгоритмов.
\end{enumerate}

При сравнимом времени выполнения эволюционный алгоритм на основе муравьиной кучи показал меньшую точность, чем генетический алгоритм: для $f(x)$ среднеквадратичная ошибка была в 1.5 раз больше, для $g(x)$ --- в 48 раз больше. Для оптимизации качества работы эволюционного алгоритма можно провести исследование с целью подбора параметров алгоритма, а также изменить критерий для отбора решений поставленной задачи.
