\chapter{Технологическая часть}

\section{Средства реализации}

В качестве языка программирования для реализации выбранных алгоритмов был выбран язык программирования Python \cite{pythonlang} ввиду наличия библиотек, реализующих генетические и эволюционные алгоритмы. Для реализации генетического алгоритма использовалась библиотека PyGAD~\cite{pygad}, для эволюционного --- ACOR~\cite{acor}.

\section{Реализация алгоритмов}

На листинге \ref{lst:1} представлена реализация генетического алгоритма аппроксимации функций.

\begin{lstlisting}[label=lst:1,caption=Генетический алгоритм аппроксимации функций]
	from math import sin, cos, pow, pi
	import numpy as np
	import pygad
	from time import process_time
	
	def f(x: float) -> float:
		return sin(x)*(sin(x)+cos(x))
	
	def g(x: float) -> float:
		return x**3 - x + 3
	
	polynomial_power = 3
	func = g
	x_start, x_end = -5, 5
	x_step = 0.001
	
	x = [i for i in np.arange(x_start, x_end + x_step/2, x_step)]
	function_inputs = [[pow(arg, polynomial_power - i) for i in range(polynomial_power + 1)] for arg in x]
	
	def fitness(_, solution, solution_idx):
		deviation = 0
		for i in range(len(x)):
			desired_output = func(x[i])
			actual_output = np.sum(solution*function_inputs[i])
			deviation += (desired_output - actual_output)**2
		return 1.0 / ((deviation/len(x))**0.5)
	
	start_time = process_time()
	ga_instance = pygad.GA(num_generations=1000,
		num_parents_mating=4,
		fitness_func=fitness,
		sol_per_pop=8,
		num_genes=polynomial_power+1,
		init_range_low=-2,
		init_range_high=5,
		parent_selection_type="sss",
		keep_parents=1,
		crossover_type="single_point",
		mutation_type="random",
		mutation_percent_genes=10)
	ga_instance.run()
	solution, solution_fitness, _ = ga_instance.best_solution()
	print("Time: {diff} seconds".format(diff = process_time() - start_time))
	print("Parameters of the best solution : {solution}".format(solution=solution))
	print("Fitness value of the best solution = {solution_fitness}".format(solution_fitness=solution_fitness))
	
	ga_instance.plot_fitness()
\end{lstlisting}

На листинге \ref{lst:2} представлена реализация эволюционного алгоритма аппроксимации функций на основе муравьиной кучи.

\begin{lstlisting}[label=lst:2,caption=Эволюционный алгоритм аппроксимации функций]
from math import sin, cos, pow, pi
import numpy as np

import Population, Acor, Constants
from time import process_time

def f(x: float) -> float:
	return sin(x)*(sin(x)+cos(x))

def g(x: float) -> float:
	return x**3 - x + 3

polynomial_power = 5
func = f
x_start, x_end = -2*pi, 2*pi
x_step = 0.001

x = [i for i in np.arange(x_start, x_end + x_step/2, x_step)]
function_inputs = [[pow(arg, polynomial_power - i) for i in range(polynomial_power + 1)] for arg in x]

def error(solution):
	deviation = 0
	for i in range(len(x)):
		desired_output = func(x[i])
		actual_output = np.sum(solution*function_inputs[i])
		deviation += (desired_output - actual_output)**2
	return (deviation/len(x))**0.5

def cost(function_input):
	return -func(function_input[0])

start_time = process_time()
acor = Acor.AcorContinuousDomain(n_pop=Constants.AcoConstants.N_POP, 
	n_vars=polynomial_power+1,
	cost_func=error,
	domain_bounds=[-20, 20])
acor.runMainLoop()
print("Time: {diff} seconds".format(diff = process_time() - start_time))
print("Parameters of the best solution : {solution}".format(solution=acor.final_best_solution.position))
print("Cost value of the best solution = {solution_cost}".format(solution_cost=acor.final_best_solution.cost_function))
\end{lstlisting}


\section*{Вывод}

В данном разделе были приведены детали реализации алгоритмов аппроксимации функций.

\clearpage
