\ssr{ВВЕДЕНИЕ}

Генетические и эволюционные алгоритмы являются разновидностью численных методов решения оптимизационных задач. Основное внимание в них уделяется использованию 
принципов естественного отбора, мутации и рекомбинации, имитирующих естественный процесс эволюции, для поиска оптимального или близкого к оптимальному решения 
сложной проблемы.

Целью данной лабораторной работы является составление программы, которая, с использованием алгоритмов оптимизации (генетического и эволюционного на основе муравьиной кучи), аппроксимирует функции $f(x) = sin(x)*(sin(x)+cos(x)), x\in [-2\pi; +2\pi], g(x) = x^3 - x +3$.

Задачи данной лабораторной работы:
\begin{enumerate}[label*=\arabic*)]
	\item описать общие этапы функционирования системы;
	\item описать предлагаемый генетический алгоритм;
	\item описать алгоритм муравьиной кучи;
	\item реализовать программу для аппроксимации заданных функций;
	\item измерить среднеквадратичную ошибку и время выполнения алгоритмов;
	\item сравнить реализованные алгоритмы.
\end{enumerate}

\clearpage
