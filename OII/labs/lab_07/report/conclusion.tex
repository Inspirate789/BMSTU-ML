\ssr{ЗАКЛЮЧЕНИЕ}

В рамках лабораторной работы была проведено сравнение алгоритмов кластеризации векторов методами K-средних и C-средних. Все поставленные задачи были выполнены.

\begin{enumerate}[label*=\arabic*)]
	\item Проведена кластеризация заданного набора данных.
	\item Определены результаты работы при различном указываемом количестве кластеров.
	\item Определено среднее внутрикластерное расстояние и среднее межкластерное расстояние для каждого рассматриваемого случая.
\end{enumerate}

Качество кластеризации текстовых данных методами K-средних и C-средних зависит от выбора метода предварительной обработки текстов, а также от близости количества выбранных кластеров к реальному. Алгоритм K-средних показал наилучшие результаты кластеризации текстовых данных, особенно при использовании лемматизации.
