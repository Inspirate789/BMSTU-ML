\chapter{Конструкторская часть}



\section{Векторизация текстов методом TF-IDF}

TF-IDF представляет собой статистический показатель, который используется для оценки важности слова в контексте документа и всего корпуса. Он состоит из следующих составляющих.
\begin{itemize}[label*=---]
	\item \textbf{TF} (Term Frequency) — частота термина в документе. Это количество раз, которое слово появляется в данном документе.
	\item \textbf{IDF} (Inverse Document Frequency) — обратная частота документа. Этот показатель уменьшает вес часто встречающихся слов в корпусе, таких как стоп-слова (например, <<и>> на <<с>>).
\end{itemize}

Итоговый показатель TF-IDF вычисляется как произведение этих двух величин:
\begin{equation}
	TF-IDF(t, d) = TF(t, d) \times IDF(t),
\end{equation}
где $t$ --- это слово, $d$ --- документ, а IDF определяется по формуле
\begin{equation}
	IDF(t)=log \frac{N}{df(t)},
\end{equation}
где $N$ --- общее количество документов в корпусе, а $df(t)$ — количество документов, содержащих слово $t$. Метод TF-IDF позволяет выделить важные слова и фразы, которые будут использоваться в качестве признаков для кластеризации.



\section{Метрики оценки качества кластеризации}

Для оценки качества кластеризации предлагается использовать следующие метрики.
\begin{itemize}[label*=---]
	\item \textbf{Среднее внутрикластерное расстояние} --- мера того, насколько объекты внутри одного кластера похожи друг на друга. Чем меньше это расстояние, тем более компактными являются кластеры.
	\item \textbf{Среднее межкластерное расстояние} --- расстояние между центроидами различных кластеров. Чем больше это расстояние, тем лучше разделены кластеры.
\end{itemize}

Данные метрики позволяют оценить, насколько качественно проведена кластеризация, и дают возможность сравнивать различные алгоритмы кластеризации.



\section{Визуализация кластеров}

Для наглядной оценки результатов кластеризации используется метод главных компонент (PCA). PCA позволяет снизить размерность данных до двух или трёх компонентов, что позволяет представить многомерные данные в двумерном или трёхмерном пространстве. С помощью этого метода можно визуализировать кластеры и оценить их разделимость. Визуализация помогает лучше понять, как алгоритм классифицирует объекты и насколько хорошо он справляется с поставленной задачей.



\clearpage
