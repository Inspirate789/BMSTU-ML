\chapter{Аналитическая часть}



\section{Кластеризация методом K-средних}

Алгоритм K-средних является одним из самых популярных методов кластеризации. Он основан на итеративном процессе, в ходе которого определяется k центроидов кластеров. В каж- дой итерации алгоритм присваивает каждому объекту (в нашем случае, каждому документу) метку, соответствующую ближайшему центроиду. Затем обновляются координаты центроидов, которые рассчитываются как среднее значение всех объектов, принадлежащих каждому кла- стеру. Процесс продолжается до тех пор, пока центроиды не перестанут изменяться, либо не будет достигнут лимит по числу итераций.
Преимущества метода K-средних:
\begin{itemize}[label*=---]
	\item подходит для случаев, когда кластеры имеют форму, приближенную к круглой.
\end{itemize}
Недостатки:
\begin{itemize}[label*=---]
	\item требуется заранее задавать количество кластеров k, что не всегда возможно без предварительного анализа данных;
	\item алгоритм чувствителен к выбору начальных центроидов, что может привести к искажению результатов кластеризации;
	\item не работает хорошо в случае сложных или эллиптических форм кластеров.
\end{itemize}



\section{Кластеризация методом C-средних}

Алгоритм Fuzzy C-Means (FCM) --- один из классических методов кластеризации. В отличие от K-Means, объекты данных в FCM могут принадлежать нескольким кластерам одновременно, причем каждому кластеру сопоставляется степень принадлежности объекта (значение из интервала [0, 1]). Этот метод широко применяется, когда данные имеют нечёткие границы или не подходят для традиционной жёсткой кластеризации.

FCM нацелен на минимизацию расстояния между данными (объектами) и центроидами кластеров, при этом каждому объекту фиксируется нечеткая степень принадлежности к каждому кластеру. В результате вместо жёсткого разделения на кластеры (как, например, в K-Means) объект может принадлежать сразу нескольким кластерам с различными коэффициентами степени принадлежности.
Преимущества метода FCM:
\begin{itemize}[label*=---]
	\item размытие границ кластеров: алгоритм может быть полезен для данных, у которых границы между группами нечёткие;
	\item наличие возможности регулирования степени размытости (нечёткости) кластеров.
\end{itemize}
Недостатки:
\begin{itemize}[label*=---]
	\item необходимость подбора большего количества параметров, чем для алгоритма K-средних;
	\item уязвимость к выбросам: малые изменения входных данных могут повлиять на местоположение центроидов кластеров;
	\item производительность как правило ниже, чем у алгоритма K-средних, по причине перерасчёта степеней принадлежности кластерам на каждой итерации.
\end{itemize}


\clearpage
