\ssr{ВВЕДЕНИЕ}

Кластеризация является одной из важнейших задач в области анализа данных. Она используется для группировки объектов в такие подмножества (кластеры), в которых объекты внутри каждого кластера максимально схожи, а объекты из разных кластеров максимально различны. Задача кластеризации широко применяется в различных областях, включая обработку текстов, анализ изображений, биоинформатику и многие другие.

Целью данной лабораторной работы является сравнение алгоритмов кластеризации векторов методами K-средних и C-средних.

Задачи данной лабораторной работы:
\begin{enumerate}[label*=\arabic*)]
	\item провести кластеризацию заданный набор данных;
	\item привести результаты работы при различном указываемом количестве кластеров;
	\item определить среднее внутрикластерное расстояние и среднее межкластерное расстояние для каждого рассматриваемого случая.
\end{enumerate}

\clearpage
