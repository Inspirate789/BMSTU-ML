\chapter{Аналитическая часть}



\section{Устройство нейронной сети}

Архитектура нейронной сети включает следующие основные компоненты.
\begin{itemize}[label*=---]
	\item \textbf{Входной слой}. Каждый нейрон этого слоя представляет одну характеристику данных. Например, для изображений из набора MNIST входной слой содержит 28 × 28 = 784 нейронов, что соответствует количеству пикселей изображения.
	\item \textbf{Скрытые слои}. Эти слои обрабатывают информацию, выделяя сложные закономерности. Их количество и конфигурация определяют способность сети к обобщению. В данной работе исследуются архитектуры с различным количеством скрытых слоев: 0, 1 и 5.
	\item \textbf{Выходной слой}. Он состоит из 10 нейронов, соответствующих числу классов в наборе данных (цифры от 0 до 9). Каждый нейрон отображает вероятность принадлежности изображения к соответствующему классу.
\end{itemize}



\section{Подготовка данных}

Для обучения нейронной сети данные должны пройти следующие этапы предварительной обработки.
\begin{itemize}[label*=---]
	\item \textbf{Нормализация данных}. Значения интенсивности пикселей преобразуются в диапазон [0, 1]. Это ускоряет обучение, устраняет проблемы численной нестабильности и делает модель менее чувствительной к масштабу данных.
	\item \textbf{One-hot encoding меток}. Метки классов преобразуются в двоичные векторы длиной 10, где значение 1 указывает на правильный класс, а остальные элементы равны 0.
\end{itemize}
Подготовка данных играет ключевую роль в повышении точности и стабильности обучения.



\section{Анализ результатов обучения}
Для анализа влияния объёма данных на эффективность классификации проводится обучение с различными соотношениями обучающей и тестовой выборок. Рассматриваются следующие сценарии.
\begin{itemize}[label*=---]
	\item \textbf{Недообучение}. Наблюдается при недостаточном объёме данных или чрезмерно простой архитектуре модели. В результате точность классификации остаётся низкой как на обучающей, так и на тестовой выборке.
	\item \textbf{Переобучение}. Происходит при слишком сложной архитектуре модели относительно объёма данных. В этом случае модель показывает высокую точность на обучающей выборке, но низкую — на тестовой, что свидетельствует о слабой обобщающей способности.
\end{itemize}
Результаты обучения позволяют выявить оптимальное соотношение данных в выборках и настроить модель для достижения баланса между точностью и обобщающей способностью.

\clearpage
