\ssr{ВВЕДЕНИЕ}

Современные достижения в области искусственного интеллекта и машинного обучения значительно расширили возможности анализа и обработки данных. Одной из ключевых задач в машинном обучении является классификация, особенно в области обработки изображений, где технологии глубокого обучения демонстрируют выдающиеся результаты.

Целью данной лабораторной работы является классификация данных из датасета MNIST с использованием нейросетевого подхода с заданными функциями активации и потерь (ReLU и Cross-Entropy соответственно).

Задачи данной лабораторной работы:
\begin{enumerate}[label*=\arabic*)]
	\item определить состояния переобучения и недообучения для различного соотношения обучающей и тестовой выборок;
	\item определить состояния переобучения и недообучения для различного количества скрытых слоёв нейронной сети;
	\item рассчитать аналитически необходимый размер обучающей выборки по неравенству Чебышёва, необходимое для гарантированного успешного выполнения поставленной задачи.
\end{enumerate}

\clearpage
