\chapter{Технологическая часть}

\section{Формирование базы знаний}

Для формирования базы знаний использовался язык SQL. На листинге \ref{tables} и \ref{data} представлены скрипты создания и заполнения базы знаний.

\begin{lstlisting}[
	caption={Создание таблиц базы знаний},
	label=tables]
create table roles (
	id bigint generated always as identity primary key,
	name text,
	design_control boolean
);

create table tasks (
	id bigint generated always as identity primary key,
	role_id bigint references roles(id) on update cascade on delete cascade,
	name text,
	certainty boolean
);

create table dev_stages (
	id bigint generated always as identity primary key,
	name text,
	content_finalized boolean
);

create table document_formats (
	id bigint generated always as identity primary key,
	name text,
	in_use boolean,
	verification_automatizable boolean,
	certainty_of_need boolean,
	necessary boolean
);

create table verification_tools (
	id bigint generated always as identity primary key,
	name text
);

create table document_formats_and_verification_tools (
	document_format_id bigint references document_formats(id) on update cascade on delete cascade,
	verification_tool_id bigint references verification_tools(id) on update cascade on delete cascade
);

create table difficult_checks (
	id bigint generated always as identity primary key,
	name text,
	reason text
);

create table regulatory_documents (
	id bigint generated always as identity primary key,
	name text,
	versatility boolean,
	concreteness int
);

create table testvkr_funtions (
	id bigint generated always as identity primary key,
	name text,
	is_primary boolean,
	is_full boolean
);
\end{lstlisting}

\begin{lstlisting}[
caption={Заполнение таблиц базы знаний},
label=data]
insert into roles(name, design_control)
values ('@студент@', true),
       ('@научный руководитель@', false),
       ('@реценент@', false),
       ('@нормоконтролер(*@', true);

insert into tasks(role_id, name, certainty)
values (1, '@писать работу@', true),
       (2, '@проверять содержание@', true),
       (3, '@проверять содержание@', true),
       (4, '@проверять оформление@', true),
       (2, '@проверять оформление@', false);

insert into dev_stages(name, content_finalized)
values ('@написание работы@', false),
       ('@проверка научным руководителем@', false),
       ('@рецензия@', true),
       ('@нормоконтроль@', true);

insert into document_formats(name, in_use, verification_automatizable, certainty_of_need, necessary)
values ('@листы бумаги с печатным текстом@', true, false, false, true),
       ('@листы бумаги с рукописным текстом@', true, false, false, false),
       ('@pdf@', true, true, true, true),
       ('@word@', false, true, false, false),
       ('@исходники tex@', false, true, false, false),
       ('@сканы@', false, true, false, false),
       ('@скриншоты электронной версии@', false, true, false, false);

insert into verification_tools(name)
values ('@глазомер@'),
       ('@линейка@'),
       ('@testvkr@');

insert into document_formats_and_verification_tools(document_format_id, verification_tool_id)
values (1, 1),
       (1, 2),
       (2, 1),
       (2, 2),
       (3, 3);

insert into difficult_checks(name, reason)
values ('@Количество абзацных отступов@', '@Это можно делать тремя различными способами: табуляциями, выравниванием или пробелами@'),
       ('@Приложения@', '@Их содержание очень разнообразно: от чертежей на А1 (ИУ1) до листингов кода (ИУ7); для унификаии принято оформлять их как картинки, но это не полное решение проблемы@'),
       ('@Заголовок "ВВЕДЕНИЕ"@', '@Студенты могут принести большое множество вариантов неправильного оформления@'),
       ('@Размер абзацного отступа@', '@Бумажная и электронная версия могут иметь разные отступы из-за ошибок при печати@'),
       ('@Маркировка списков@', '@Списки могут быть вложенными@'),
       ('@Нумерация рисунков и таблиц@', '@Может быть как в рамках одного раздела, так и сквозная по всему документу@'),
       ('@Ссылки на источники@', '@Можно перепутать с обозначением отрезка@'),
       ('@Размер абзацного отступа@', '@Бумажная и электронная версия могут иметь разные отступы из-за ошибок при печати@');

insert into regulatory_documents(name, versatility, concreteness)
values ('@ГОСТ@', true, 0),
       ('@приложения к ГОСТ для ВУЗ@', false, 1),
       ('@требования кафедры@', false, 2);

insert into testvkr_funtions(name, is_primary, is_full)
values ('@проверка плагиата@', true, true),
       ('@загрузка в БД антиплагиата@', true, true),
       ('@проверка наличия разделов@', false, true),
       ('@проверка корректости оформления структурных элементов@', false, false);
\end{lstlisting}

\section{Запросы к системе}

Для получения данных из базы знаний запросы должны быть переведены с естественного языка на язык SQL. Ниже представлены примеры запросов к системе.

\begin{enumerate}[label*=\arabic*.]
	\item Какие функции научного руководителя являются необязательными? (2 параметра)
	\begin{lstlisting}
	postgres=# select t.name 
	           from roles r inner join tasks t on r.id = t.role_id 
	           where r.name = '@научный@ @руководитель@' 
	                 and t.certainty = false;
	@проверять оформление@
	\end{lstlisting}
	\item Может ли научный руководитель проверять оформление РПЗ студента? (2 параметра)
	\begin{lstlisting}
		postgres=# select exists(
		               select 1
		               from roles r inner join tasks t on r.id = t.role_id 
		               where r.name = '@научный руководитель@' 
		                     and t.name = '@проверять оформление@'
		);
		t
	\end{lstlisting}
	\item Правда ли, что всё ещё можно представлять РПЗ в рукописном виде? (1 параметр)
	\begin{lstlisting}
		postgres=# select in_use 
		           from document_formats 
		           where name = '@листы бумаги с рукописным текстом@';
		t
	\end{lstlisting}
	\item Почему проверка количества абзацных отступов считается сложной? (1 параметр)
	\begin{lstlisting}
		postgres=# select reason 
		           from difficult_checks 
		           where name = '@Количество абзацных отступов@';
		t
	\end{lstlisting}
	\item Каковы основные назначения testvkr? (2 параметра)
	\begin{lstlisting}
		postgres=# select name 
		           from testvkr_funtions 
		           where is_primary = true and is_full = true;
		@проверка плагиата@
		@загрузка в БД антиплагиата@
	\end{lstlisting}
\end{enumerate}

\clearpage
