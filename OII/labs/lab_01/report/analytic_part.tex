\chapter{План опроса эксперта}

Тема: особенности проверки текстов РПЗ на соответствие ГОСТ и автоматизация этого процесса.

Эксперт: Мальцева Диана Юрьевна.

\section{Материалы для ознакомления}
\begin{enumerate}[label*=\arabic*)]
	\item ГОСТ 7.32-2017 (<<Отчёт о научно-исследовательской работе>>);
	\item инструкция по оформлению ВКР и проверке на объем заимствования для студентов МГТУ им. Н.Э. Баумана; 
	\item памятка <<Как написать для руководителя заготовку задания на дипломную работу>>; 
	\item приложение 1 к положению о нормоконтроле (инструкция по работе с электронно-библиотечной системой <<Банк ВКР>>). 
\end{enumerate}
Все материалы представлены в приложениях к отчёту.

\section{Вопросы эксперту}

\begin{enumerate}[label*=\arabic*.]
	\item Какая мотивация к автоматизации процесса проверки РПЗ у разных сторон (студентов, научных руководителей, нормоконтролеров)?
	
	\item Какие РПЗ подлежат проверке?
	\begin{enumerate}[label*=\arabic*.]
		\item Файлы каких форматов необходимо проверять?
		\item Почему pdf --- устоявшийся стандарт?
		\item Какие можно рассмотреть альтернативы pdf?
	\end{enumerate}
	
	\item Какие есть различия между кафедрами и специальностями в требованиях к оформлению РПЗ ВКР? 
	\begin{enumerate}[label*=\arabic*.]
		\item Всем ли нужны одни и те же ГОСТ?
		\item Насколько сильно ГОСТ и дополнительные требования кафедр ограничивают вариативность при оформлении РПЗ?
		\item Насколько различным может быть оформление различных РПЗ, одинаково удовлетворяющих требованиям? В каких аспектах оформления эти различия могут выражаться?
	\end{enumerate}
	
	\item Возможно ли разработать ПО под нужды каждой заинтересованной стороны? Если нет, то интересы какой стороны процесса следует учесть в первую очередь?
	
	\item На данный момент можно представить 2 способа проверки оформления РПЗ: по готовому документу (альтернатива ручной проверки с линейкой в реальной жизни) и по вёрстке.
	\begin{enumerate}[label*=\arabic*.]
		\item Есть ли ещё какие-либо подходы к проверке РПЗ?
		\item Какой из них в наибольшей степени покрывает исходные требования к оформлению РПЗ?
		\item Какой из них в наибольшей степени поддаётся автоматизации?
	\end{enumerate}
	
	\item Возможно ли полностью автоматизировать процесс проверки РПЗ? Если нет, то до какой степени всё же можно это сделать?
	
	\item Какой сейчас достигнут прогресс по автоматизации?
	
	\item Сейчас существует testvkr в виде программы и памяток. Стоит ли развивать этот проект дальше?
	\begin{enumerate}[label*=\arabic*.]
		\item Если да, то в каком направлении?
		\item Если нет, то каким (в общих чертах) должно быть решение проблемы?
		\item Какие есть наработки/идеи помимо testvkr?
		\item О каких подводных камнях реализации автоматической проверки РПЗ уже известно?
		\item С какими трудностями при реализации возможно предстоит столкнуться?
	\end{enumerate}
\end{enumerate}



%<пишем про тему и все, связанное с ней>

%\section*{Вывод}
%
%<что сделали в аналите, кратко>

\clearpage
