\chapter{Опрос эксперта}

Сокращения: <<\textbf{Э}>> --- эксперт, <<\textbf{И}>> --- интервьювер.

\textbf{И: }Какая мотивация у всех сторон процесса [проверки текстов РПЗ на соответствие ГОСТ]? Есть студенты, которые хотят <<пропихнуть>> свою работу, есть научные руководители, которые формально руководят этим процессом. Они же должны перед нормоконтролем подписать работу.

\textbf{Э: }Научные руководители смотрят работу с точки зрения, о чём она. Они смотрят текст. Сложно сказать, входит ли у них в должностную инструкцию проверка этого всего на ГОСТ. Они должны проверить с точки зрения работы. Опять-таки, они сопровождают работу, они её не пишут. То есть дать на определённых этапах определённые рекомендации и получить от студента ответы на определённые вопросы --- это их задачи. А выверять, правильно ли нарисована схемы или правильно ли структурирован по абзацам и по разделам, они в общем-то не сильно обязаны. 

\textbf{И: }Так, зафиксируем, что им в большинстве случаев всё равно, хотя они могли бы регулировать этот процесс.

\textbf{Э: }Скажем так, не то чтобы всё равно, у них есть чёткое разграничение, что не все области актуальности и аналитической области работ совпадают с той областью, в которой работает конкретный руководитель, поэтому он не полезет область, в которой он не совсем всё знает, поэтому сопроводить формально работу он может: ему достаёт квалификации и знаний, чтобы знать какие-то общие моменты о данной работе в данном направлении, но у него нет конкретных специфических знаний. Он же не будет проделывать работу за студента, искать те же методы, на которых базируется его работа, чтобы повторно потом её переделывать.

\textbf{И: }Так, с научными руководителями теперь стало понятнее. Есть ещё третья сторона --- нормоконтролер (вроде бы последняя), которая должна разгребать всё, что осталось, с точки зрения оформления.

\textbf{Э: }Если мы говорим о работе в целом, то у нас есть несколько аспектов. Во первых, кафедра в лице научного руководителя выдаёт студенту тему. Студент начинает в этой теме разбираться: в начале на этапе каких-то мелких работ или же сразу на уровне выпускной работы. Далее они идут вместе: студент что-то пишет, руководитель каким-то образом проверяет, что тот пишет. НУ хотя бы, условно, хотя бы даже по отсечкам, по количеству написанных разделов на определённую дату. Отдельно есть процесс рецензирования. Рецензии делает либо преподаватель кафедры, либо преподаватель какой-либо сторонней кафедры, то есть человек, который так или иначе задействован области, с сопряжённой предметной областью работает, который может оценить то, насколько это работа актуальна, насколько она корректно выполнена.

\textbf{И: }Под корректностью мы здесь подразумеваем не корректное оформление?

\textbf{Э: }Не оформление, а именно с точки зрения содержания. То есть рецензенту предоставляется уже какой-то более-менее готовый текст, под которым он подписывается (под содержанием работы). И соответственно, у нас параллельно идёт, грубо говоря, работа со смыслом работы (у научного руководителя) рецензии и прочие формальные вещи и есть тот, кто работает с оформлением. По идее научный руководитель может влезать в работу с оформлением, но они будут делать это не всегда, потому что это не их задача. Задача правильно оформить работу --- задача самого студента. Соответственно нормоконтроль и антиплагиат --- тоже параллельные процессы. Это по сути не один акт, это по сути два акта, в общем-то и делать это могут два разных человека. Только обязанности все свели одному, соответственно отдельно антиплагиат работает именно по смысловому её содержанию, отдельно нормоконтроль именно на соответствие правильности оформления по ГОСТ.

\textbf{И: }Хорошо, Теперь стало понятнее, что главный ответственный за проверку здесь нормоконтролер. 

\textbf{Э: }Ну, именно ГОСТ, именно формального оформление.

\textbf{И: }Давайте сформулируем, какие у нас вообще есть РПЗ, что на входе? Все же кафедры, могут что-то по-разному требовать, где-то требования пожёстче, где-то послабее, ГОСТ хоть одни и те же, но сами входные данные разные. Мы условно можем, принимать [форматы файлов с РПЗ] docx и pdf...

\textbf{Э: }Тут не столько дело во входных данных...

\textbf{И: }Нам сейчас просто для дальнейших вопросов важно очертить формат входа. Он вообще какой может быть? Есть ли смысл вообще как-то ограничивать его?

\textbf{Э: }По-хорошему, нормоконтролер получает два варианта работы. Он мог бы вообще получать один формат работы --- бумажный А электронный вариант ещё берётся, потому что у нас происходит автоматическая проверка на плагиат, и он грузится в систему. И вот тут возникает вопрос в том, что нам нужно сверить, что у нас в электронном варианте и в физическом одна и та же работа, одно и тоже оформление, один и тот же текст.

\textbf{И: }Так я помню, что по крайней мере в годы, когда я сдавал диплом и курсовую по БД, это принималось на веру. 

\textbf{Э: }С БД не могу сказать, с ВКР (с учётом того, что этим занималась я) мы сверяли. Сверяли текст электронный и физический.

\textbf{И: }Может быть, я что-то забыл, извиняюсь...

\textbf{Э: }Вероятно, вы могли пропустить. Мы всегда смотрели две работы и смотрели текст параллельно. Помимо того, что мы сверяем два варианта, всё равно осматриваем мы в основном физический, потому что в электронном виде, конечно, что-то отловить проще БЫЛО, но у нас нет средств автоматизации. Поэтому меряем с линейкой и карандашом. Отсюда, как бы, если мы говорим именно о факторе, когда у нас проверка на нормоконтроль  выполняется человеком, то тут как бы 200 листов и вперёд. Если мы грузим это через какую-то систему, то тут надо понимать, что должна быть система распознавания отдельных вещей. Потому что, допустим то, что грузится в антиплагиат проверялось на регистр разделов. Там прикручены определённые вещи, которые может проверить именно система. То есть, там что-то базовое: наличие всех разделов, то, что все разделы на месте. Отдельно могла быть проверка на заполненность этих разделов по количеству слов, подписи к рисункам не проверялись, отдельно могло быть установлено наличие разделов и их заполненность, проверяются все структурные элементы всех разделов. Более никакие вещи из оформления оно не отлавливало, то есть оно не анализировала текст. Из оформления то, на чём все спотыкались, --- это то, что невозможно отследить программными средствами. Невозможно отследить по электронной копии, допустим, количество, отступов от края листа. Его сложно отследить за счёт того, что один делает это грубо говоря табуляциями, один это делает выравниванием, другой делает это пробелами. Это три варианта разных отступов.

\textbf{И: }Имеет ли смысл здесь для автоматизации ограничить проверку не форматом pdf, не docx, не исходниками TeX, А сканами. Есть ли смысл пытаться делать так, чтобы действительно по пикселям считать, как по линейке, как вы это делаете? А электронная копия останется.

\textbf{Э: }Так, здесь подразумевается, что мы берём наши 200 листов, пихаем их в сканер, получаем сканы и начинаем дальше обрабатывать. Или же мы можем сделать скан из электронной версии

\textbf{И: }Скан электронной версии я как-то себе не очень представляю. Так что речь идёт про скан из бумаги. Да, минус в том, что это тяжело делается. Но если мы каким-то образом это достали (например, заставили студента), то, наверное, будет легче повторить вашу же работу. Именно так, как вы её делаете. Есть ли в этом смысл вообще?

\textbf{Э: }Сложно сказать, на самом деле, если в этом смысле, то есть вообще, в принципе, сложно сказать, есть ли смысл перегонять физическую версию назад в электронным путём уже сканирования, будет ли это грубо говоря прохождение одних и тех же кругов, которые можно было бы избежать, или же это даёт в перспективе какой-то больший плюс, чем минус.

\textbf{И: }Можно же, грубо говоря, электронную версию поскринить, и это будет по сути что-то типа того же скана, это будет фотография из пикселей, по которым можно нормально мерить расстояние, подогнать формат кадра. и вот это мы уже можем сделать с небольшими потерями нервов и времени. Есть ли смысл делать так?

\textbf{Э: }Сложно сказать.

\textbf{И: }Почему это может быть тяжело с точки зрения автоматизации? Мы будем стараться повторять вашу работу, то есть мерить те же расстояния, считать количество тех же отступов.

\textbf{Э: }200 человек по 200 листов, все защиты укомплектованы в 1 месяц работы. Это буквально, в среднем 4 заседания в неделю. Соответственно 4 недели по четыре заседания в неделю (временами даже 5), 4 на 4 на 8 человек, итого 128 чистыми, бакалавров плюс магистров еще 6 свиданий, итого, укладываемся с округлением в 200 человек. В среднем работа бакалавра --- от 60 листов, работа магистра --- около 100-120, 130. Усредняем, берём 150 листов. Итого 30000 листов (сканов). То есть условно 200 работ по 150 листов, которые надо отсканировать. Нормоконтроли начинаются примерно за неделю до защиты. В идеале, конечно, за 2 недели. Начинать нормоконтроль раньше 27, по-моему, мая в прошлом году было, например, нельзя, потому что только 27 мая студенты официально выпускаются с практики преддипломной. Соответственно, вся проверка начинается, грубо говоря, параллельно с работой комиссии со сдвигом в неделю. соответственно, у нас те же сжатые сроки, те же 4 недели.

\textbf{И: }Получается, в текущих условиях вы вместе с магистрами точно также обрабатываете руками эти работы, но программа же будет работать, наверное, быстрее...

\textbf{Э: }Будет.

\textbf{И: }При этом нормоконтроллер не будет тратить время на то, чтобы сделать эти страны, Пускай приносят студенты.

\textbf{Э: }Тем не менее тогда придётся проверять отсканированную, печатную работу и электронную версию. То есть, всё равно сверять все 3 работы, что это действительно одна и та же работа, а не три разных.

\textbf{И: }То есть от проверки бумаги уйти никак нельзя...

\textbf{Э: }Ну чисто теоретически можно, если мы начнём проверять нормоконтроль на электронной версии. Проблема в том, что работает всё равно сдаётся печатном виде. 

\textbf{И: }Поэтому бумагу всё равно приходится проверять?

\textbf{Э: }Да, потому что могут быть дефекты печати, дефекты, грубо говоря, изменения печатного варианта от электронного, то есть вот эти все вещи.

\textbf{И: }Так, получается, от бумаги никак не уйти, и всё равно никак не уйти от сверки. На всякий случай, когда-то же дипломы писались руками лет 20 там с лишним назад. А до сих пор формально можно ли так сдавать? Я не в одних регламентах такого не видел упоминания, вообще этой вещи

\textbf{Э: }Написанный от руки диплом?

\textbf{И: }Да.

\textbf{Э: }Так, чисто теоретически, если при этом соблюдаются все формальные отступы и...

\textbf{И: }... и шрифт Comic Sans и я написал им всё аккуратно.

\textbf{Э: }Comic Sans... ну нет, нужен какой-то читаемый шрифт с засечками. Comic Sans это что-то из весёлого. Ну, нет, если она читаемыми печатными буквами написано... А хотя вообще, кстати, хороший вопрос на засыпку

\textbf{И: }Я ещё когда сам диплом писал, ради интереса смотрел, можно ли руками написать. 

\textbf{Э: }Проблема в том, что это действительно раньше писалось так и, насколько я помню, прямого запрета нигде не выпускалось. 

\textbf{И: }Вот так я и не видел сам. Давайте в дальнейшем диалоге тогда это держать в голове, вдруг это поможет чем-то или лишнее ограничение будет.

\textbf{Э: }Мне нужно найти одного сумасшедшего, который принесёт диплом, написанный от руки и порадует этим комиссию. Где найти одного безумца? Ну ладно.

\textbf{И: }Ну вам же на дипломы кого-то дают...

\textbf{Э: }Ну что я зверь, что ли. Мне нужен другой безумец, у которого не я буду научным руководителем. Я не хочу быть к этому причастной, я хочу это наблюдать. Я то почитаю, какая разница, будет фотографии работы присылать.

\textbf{И: }Хорошо, давайте вернёмся к требованиям. Во-первых, заметны ли сильные различия между кафедрами разными по требованиям к РПЗ ВКР?

\textbf{Э: }Ну как различия... ГОСТ у всех единый.

\textbf{И: }То есть всем нужен один и тот же ГОСТ, одна и та же версия?

\textbf{Э: }Оформление может различаться от учебного заведения к учебному заведению, какой-то регламент оформления работ в рамках ВУЗа. В нашем случае...

\textbf{И: }Мы говорим только про бауманку сейчас.

\textbf{Э: }То, что касается названия кафедры, названия факультета, то, как размещается герб, потому что в разных ВУЗах он размещается по-разному. Дальше в рамках кафедр у нас идёт изменение касательно того, в чём вообще основная цель работы. У нас это метод, который сопровождается в качестве, грубо говоря, апробации доказательной базы --- это программная реализация. В то тоже время, допустим, если мы возьмём какой-нибудь диплом кафедры проектирования, не могу с ходу сказать, на каком факультете это можно искать. Да может быть, даже где-нибудь на кафедрах ИУ1, ИУ2. У них в дипломе идёт проектирование какого-либо, допустим, оборудования, техники, Соответственно, у них нет программной реализации. У них нет листингов кода, у них, возможно, нет блок-схем. Или же они есть, но используются для какой-то другой нотации. При этом у них будет чертёж на А1, чертёж на А3, чертёж ещё на чём-то плакатном. И это тоже подшивается в работу. Ну как бы оно регламентируется своими правилами. Есть регламент, грубо говоря, того, как мы заливаем листинг. Это то, что специфика нашей кафедры. При том, кстати, если порыться, то нет как таковых правил о том, как оформляется листинг с точки зрения, допустим, ГОСТ. Нету. В целом, листинг --- это картинка. По сути, единственное, что можно привязать к листингу, к его формальному представлению, это картинка...

\textbf{И: }Которую особо не поревьювишь...

\textbf{Э: }Потому что по сути код --- это картинка. Да, единственное --- подпись к ней регламентирована снизу, поэтому листинг тоже надо подписывать снизу. На кафедре мы приняли единственное решение, что мы не подписываем листинг снизу, мы подписываем листинг сверху. Просто, чтобы отличать их от картинок, потому что в том числе есть, допустим, вещь, когда в рамках разрабатываемого метода тогда идёт какая-нибудь транскрипция из одного кода в другой, ещё какая-нибудь ерунда... И картинка с кодом будет действительно картинкой, иллюстрирующей там, конечный или начальный результат работы программного обеспечения.

\textbf{И: }В таком случае, я так понимаю, мы говорим здесь о том, что к приложениям мы не предъявляем какие-то жёсткие требования и не влезаем в формат этих <<картинок>>. Не проверяем на ГОСТ чертежи ИУ1.

\textbf{Э: }Ну условно. Это примерно, как законодательство. У нас есть законодательство страны, есть законодательство на федеральный округ, есть законодательство на край или область. Край и ли область могут ставить какие-то определённые ограничения, которые будут жёстче, чем Федеральное Законодательство, но при этом будут соблюдаться только в рамках края.

\textbf{И: }Но при этом мы можем по сути привести всё единому федеральному законодательству, оно будет просто не очень-то и жёсткое?

\textbf{Э: }Не можем, если мы на момент нахождения вот здесь, мы не можем выйти за рамки того законодательства, которое ставит край или область.

\textbf{И: }Обобщить у нас нет возможности, нельзя это сделать? Есть ли противоречия между властями?

\textbf{Э: }Все противоречия, которые возникают, они всё равно сводятся к тому, на территории, грубо говоря, какой области вы сейчас варитесь.

\textbf{И: }А, если федеральные требования не будут как-то обращать внимания на эти противоречащие вещи и никак их не фиксировать, то тогда проверка слишком ослабнет или в ней всё-таки останется смысл?

\textbf{Э: }Нет, мы берём по самым жёстким, грубо говоря. То есть, у нас есть общее положение ГОСТ. ГОСТ регламентирует, как в принципе пишется техническая документация, он не говорит о конкретном...

\textbf{И: }Там в ГОСТ ещё очень часто встречаются формулировки типа <<следует ...>>, <<рекомендуется ...>>, то есть они не жёсткие.

\textbf{Э: }Да, поэтому разрабатывается помимо ГОСТ приложение, которое выпускается для ВУЗа и регламентирует работы в рамках ВУЗа, всего. Опять-таки, регламентировать все работы в ВУЗе по одному шаблону у нас не получится, хотя бы даже потому что вот у нас, допустим, на кафедре чертежей на А3, А1 нету. Нам нечего очертить. Вот, а кому-то наоборот ничего писать в листинг, но есть, что чертить. Поэтому вот здесь выпускаются свои, грубо говоря, ещё настройки на всё это, которые регламентируют отдельные какие-то запчасти. В нашем случае, допустим, листинги, в случае кого-то других чертежи. Но с чертежами попроще: чертежи начались раньше и на чертежи есть отдельные ГОСТ, поэтому у них чуть больше, скажем так, в этом плане... регламент.

\textbf{И: }Условно, если говорить именно про нашу кафедру или в целом про какую-либо отдельно взятую, то там можно взять, условно, приложение, которое сделано только для нашей кафедры и регламентируют всё жёстко по ним.

\textbf{Э: }Да, в этом и смысл.

\textbf{И: }Хорошо, то есть к единому стандарту мы это не приведём...

\textbf{Э: }К единому приведём, к универсальному не придём.

\textbf{И: }Так, мы ещё с вами говорили про способы проверки РПЗ. То есть мы это можем проверять по готовому тексту, можем делать по какой-то вёрстке. Имеет ли смысл пытаться всё-таки работать с вёрсткой? Отдельно делать валидатор для исходников LaTeX, если кто-то использует исходники LaTeX, если кто-то принёс pdf с вёрсткой. И неизвестно, где вообще pdf была собрана. Там, может, использовались разные компиляторы. Или это может быть формат Word 1997 года. Есть ли смысл проверять разные форматы?

\textbf{Э: }Вот поэтому это всё сводится либо к какому-то Word как к распространённому ПО (до недавних событий), либо к PDF как к наиболее, ну скажем, статичному из возможных, то есть при открытии его на разных устройствах у нас хотя бы вёрстка не едет, мы документ видим более-менее одинаково.

\textbf{И: }Ну раз разные viewer'ы смогли одинаково отображать PDF, то можем мы так же одинакового валидировать по этим PDF, условно, отступы или что-то типа того, как делается в testvkr.

\textbf{Э: }Я вам честно говоря скажу, что в зависимости от настроек PDF разные viewer'ы могут по-разному отображать одну и ту же PDF. Это считается, конечно, преимуществом PDF, что мы его типа открываем на разных устройствах и он выглядит одинаково... но нет. Если докопаться и найти определённую PDF, в которой не указаны определённые строгие настройки или там, грубо говоря, настройки привязаны к разрешению экрана или чему-нибудь ещё, она будет отображаться по-разному, поэтому это тоже не универсальный вариант.

\textbf{И: }Так, есть ли какие-то универсальные варианты тогда вообще?

\textbf{Э: }Да кто ж их знает то... Универсальные варианты...

\textbf{И: }Хорошо, ладно. Тогда какой вариант в большей степени вообще поддаётся автоматизации, то есть, ну, почему-то же люди, когда разрабатывали testvkr, они решили взять PDF.

\textbf{Э: }А я откуда знаю, что ещё меня спрашиваете, я эксперт в области того, как всё это проверить на нормоконтроль, а не того, как это записать в электронном виде.

\textbf{И: }Так, давайте поконкретнее к программам для автоматизации. У нас есть testvkr, которые мы все пользуемся. Есть ещё какие-то альтернативы, близкие по регламентам, которые они проверяют.

\textbf{Э: }Нет, вот так не скажу. Можно, скорее всего с какими-то определёнными настройками поставить распознавание, допустим, отдельных структурных элементов: грубо говоря, чтобы у нас из общего документа находились, допустим, картинки и находились подписи к этим картинкам в рамках допустим 1 страницы у нас есть картинка, у нас есть подпись к ней, есть таблица --- есть подпись к ней есть, есть схема, есть формулы. Ну как бы это всё на уровне компьютерного зрения, но, грубо говоря, распознавание на изображении, распознавание в документе.

\textbf{И: }Давайте ещё вернёмся к testvkr, чуть-чуть я ещё про него поспрашиваю. Вообще, как жили до него? Всё было ручками?

\textbf{Э: }Подозреваю, что да. Ну, как бы testvkr в первую очередь заводился именно для антиплагиата, то есть, соответственно, чтобы было проще проверять антиплагиат, его настраивали уже, чтобы ну что хоть какую-то минимальную вёрстку он смотрел.

\textbf{И: }То есть изначальное его предназначение вообще не в проверке оформления?

\textbf{Э: }Вообще нет. Я же говорю, он поэтому может посчитать структурные элементы: что у нас есть введение, что оно на месте, что есть аналитический раздел, что содержание, что все разделы, которые есть в содержании, они как-то находятся в документе. Вот это он может отследить. Он отслеживает, что работа правильно подписана с правильным титульным листом, чтобы у нас было соответствие с фамилией, считанной с титульника, потому что на самом деле при вёрстке акта, фамилия, конечно, берётся из базы для того студента, для которого мы грузим, но тема считывается/распознаётся с документа, который мы грузим. По этому в теме очень часто приходится править при вёрстке акта пробелы и ещё какие-то вещи. Собственно, поэтому testvkr для того, чтобы проверить плагиат. Он для того, чтобы загрузить в базу, посмотреть плагиат и вернусь назад, поэтому он может отловить такие крупные структурные элементы, потом в него вносились какие-то настройки более узкие, специализированные, но он всё ещё проверяет именно в общем, что у нас документ пришёл целый, не битый, у него все разделы. Всё. И в нём есть текст.

\textbf{И: }Как вам кажется со стороны проверяющего, есть ли смысл брать именно этот проект и пытаться развивать его дальше с точки зрения автоматизации проверки оформления, если не брать в расчёт проверку на плагиат. Ну то есть с тем, чтобы задачи загрузить базу, чтобы потом проверить плагиат, testvkr справляется, а с оформлением? Есть смысл развивать его дальше? 

\textbf{Э: }Ну, вероятно, стоит. Ну во-первых, у нас всё движется в сторону цифровизации, то есть дело времени, когда у нас будут сдаваться полностью цифровые дипломы. Мне правда хочется верить, что это будет, что я до этого ещё доживу, что это случится не после того, как я перестану иметь к этому какое-либо отношение. Вот, потому что всё движется к тому, что у нас идёт цифровизация. Соответственно, если у нас полностью цифровой диплом, ну смысл полностью цифровой диплом отсматривать глазами и руками, если это можно автоматизировать?

\textbf{И: }Так, то есть нам есть смысл двигаться в сторону полной проверки цифровой копии и про сам testvkr вопрос про развитие был не в том, стоит ли вообще его развивать, а в том, стоит ли развивать именно его? То есть он же изначально предназначался не для проверки оформления. Если я хочу валидировать именно на соответствие ГОСТ, то... 

\textbf{Э: }Возможно стоит взять отдельное программное обеспечение, которое будет это делать.

\textbf{И: }Так хорошо. То есть, есть смысл что-то своё, так ведь?

\textbf{Э: }Вполне да.

\textbf{И: }Так, аналогов у нас нет. В общих чертах, чуть подытоживая, каким должно быть решение проблемы, если мы его делаем самостоятельно для валидации соответствия ГОСТ? То есть, нам нужно, во-первых, определиться с форматом --- взять какой-то... либо картинку опять же, чтобы пытаться играть в компьютерное зрение.

*Эксперта отвлекли.*

\textbf{И: }Так, имеет смысл взять что-то типа скана, чтобы играть в компьютерное зрение. Против этого вы по-моему ничего не сказали.

\textbf{Э: }Ну тут надо смотреть и надо пробовать, на сколько это всё будет оправдано. Возможно на первоначальном этапе это будут сканы. Возможно, потом можно будет уйти от сканов в пользу, я не знаю, того же PDF. Только распознавать его именно пытаться не как набор текста, а именно картинку. Тут, скажем так, сторона, с которой в проблему зайти, --- это весьма условная вещь. А вот как оно внутри будет обрабатываться и как потом это потом будет модернизироваться --- это уже другая.

\textbf{И: }Да, здесь скорее надо пробовать, потому что пока не особо много наработано, помимо testvkr. Так, определились с форматом, пытаемся распознавать. Вопрос по самим структурным элементам, потому что мы с вами сейчас говорили только там про какие-то там проверку отступов, подсчёт количества там, возможно что-то типа линеечкой померить. Есть какие-то вещи, которые очень сложно валидировать, даже с линеечкой? Что мы вообще такого должны такое проверять, что сложнее количества и размера отступов например, или наличия чего-либо? Вот я назвал 3 категории. Есть ли что-то ещё посложнее, чего я не назвал?

\textbf{Э: }Формат, в принципе, титульного листа отсматривает и testvkr, ну то есть, грубо говоря, он смотрит наличие полей, наличие структурных элементов в том плане, что там кафедра, название ВУЗа название университета и так далее. Потом идёт, если мы пропускаем ТЗ, которое вообще идёт отдельно, идёт введение... вру, реферат, заголовок. Соответственно, все заголовки разделов нумерованные и начинаются с заглавной буквы, остальные буквы строчные. Структурные элементы <<введение>>, <<реферат>>, <<заключение>>, <<список использованных источников>> и <<приложения>>, допустим, пишутся только caps'ом. То есть, отловить текст, что у нас где-то написано слово <<введение>>, мы можем. Можем ли мы отловить, каким именно шрифтом, каким именно размером написаны эти буквы?...

\textbf{И: }Сложность здесь зависит от реализации самой, но вы считаете, что это может быть подводным камнем?

\textbf{Э: }Из того, что я могу сказать, какую, грубо говоря, гадость, студенты приносят ежегодно. Вот это самое триклятое введение, которое расположено, допустим, не посередине листа, а сдвинуто, выровнено по левому краю вместо центра, которое написано маленькими буквами вместо заглавной, которая почему-то написана с номером. Ну, с номером мы его ещё отловим, допустим, на testvkr. А вот всё остальное, допустим, уже отлавливается только руками. Потом абзацный отступ 1.25, 1.27. В общем случае он отлавливается программно. Ну то есть, это реализовано в testvkr. Возможно, есть возможность это реализовать как-то ещё программно. Это не всегда видно, допустим на бумаге, если у нас принтер выставлен с масштабом или любит жевать бумагу, исключительно по правому краю. Соответственно, у него будет сдвинута печать не на 1.27, не на 1.25. Следующее: пункты, списки. С точки зрения программного подхода. Ну то есть, ему что точка, что тирешечка, --- глубоко фиолетово, одинаково. Ну, буллит и буллит. С точки зрения визуального мы видим разницу. У нас есть принципиальная разница в том, какой из них вставлять, в каком порядке. Отдельно такая классная штука, как одновременное соблюдение правил русского языка и оформления списков. Мы же можем каждый абзац, грубо говоря, под маркером ставить с точкой или с точкой с запятой. В зависимости от этого у нас оформляется то, с большой буквы он начинается или с маленькой, можно ли перед всем этим списком ставить двоеточие или нет, и как его нужно завершать: тоже точкой или точкой с запятой и последующим каким-то предложением.

\textbf{И: }Ну короче важно позаботиться о наличии конкретных шаблонов реализации этих случаев. 

\textbf{Э: }То есть, есть вот эти случаи, причём их буквально есть два: мы знаем, что у нас если у нас вот так, то у нас большая буква и, грубо говоря, точка, а если вот так, то точка с запятой и маленькая буква. И они оба легитимны, их оба можно использовать, главное не скрещивать.

\textbf{И: }Радует, что их всего два.

*Смех эксперта сквозь боль*

\textbf{Э: }Ну их то, конечно, два...

\textbf{И: }Не понял. 

\textbf{Э: }У вас может быть вложенный список, например. У вас получится, что у вас одновременно смешано оформление и не смешано, потому что они разные по уровням.

\textbf{И: }Наверное, с этим можно жить, но так или иначе да, подводным камнем может стать. Ещё возможным подводным камнем реализации, если мы будем делать сами с нуля, может стать сам Flow работы нормоконтролера со всем ПО, потому что, допустим, когда мы начали говорить про сканы, вы спросили, будет ли сам нормоконтроллер сканить? Ну очевидно, что нет. Да, об этом надо будет ещё позаботиться. И можно здесь облажаться случайно.

\textbf{Э: }Если мы сделаем маленькое лирическое отступление, допустим, в работу той же отборочной комиссии, по которой писались в прошлом году какие-то проекты и предметные области, то там у нас есть такая замечательная штука, как электронная база, в которую абитуриенты присылают свои документы. Вероятно, вы когда поступали повторно, столкнулись. И параллельно с этим существует вариант того, что абитуриент приходит очно и подаёт документы очно. У нас с этого года было введено электронное личное дело. Соответственно, все бумажки, которые он принёс очно, они сканировались и попадали в систему в качестве сканов. То есть, либо абитуриентом самостоятельно, когда он подавал электронную копию, либо работниками отборочной [комиссии], когда они это делали сами. То есть, поэтому заставить нормоконтролера сканировать столько страниц... Я уже ничему не удивлюсь...

\textbf{И: }Хорошо, об этом надо позаботятся. Какие ещё возможные подводные камни мы могли забыть здесь, если сами с нуля делаем автоматизацию проверки соответствия РПЗ на ГОСТ?

\textbf{Э: }Ну, выравнивание таблиц, нумерация этих таблиц, нумерация рисунков, нумерация всех структурных элементов. То есть, есть вариант, когда мы делаем нумерацию сквозную через весь документ и у нас все объекты нумеруются от первого до последнего, есть вариант, когда мы планируем их по разделам, то есть сначала идёт номер раздела, потом номер самого элемента.

\textbf{И: }А разве ГОСТ допускает оба варианта?

\textbf{Э: }... Но не в рамках одного документа: либо всё полностью сквозным, либо всё по разделам, но не два вместе в одном документе. Ссылки в тексте. Ссылки в тексте на источники литературы это отдельная песня. Их вот кстати testvkr тоже пытается отлавливать. У нас есть вариант оформления в квадратных скобках ссылок на литературу, есть вариант оформления в квадратных скобках интервалов: интервалов для графиков, интервалов значений, диапазонов. С точки зрения, если мы посимвольно пытаемся читать, допустим, такую строку и посмотреть, что в ней, это одно и то же; с точки зрения визуальной, если просто глядеть на печатный лист и видеть квадратные скобки, это тоже может быть одинаково. С точки зрения, если почитать и вникнуть в смысл, вот тут как бы найдём разницу, где интервал, а где...

\textbf{И: }Здесь приходится только читать и вникать в смысл. Ну, грубо говоря, смотреть какие-то паттерны, встречающиеся рядом. Допустим, у меня...

\textbf{Э: }Ну вот, скажем, в этом году как проводилось: testvkr выкидывает, что у нас есть какие-то неопознанные ссылки, которых нет в источниках литературы. Если мы смотрим, что по номерам диапазон от 100 до 200, до 1000. Ну, понятное дело, что у нас нет такого количества источников. Понятно, что это диапазон, и можно это игнорировать. А когда у нас диапазоны маленькие: допустим, 1, 3 и целые положительные числа, то есть оно явно может быть перепутано, то, грубо говоря, открываем эту страницу, смотрим страницу и либо пропускаем, либо нет.

\textbf{И: }Помню, я с этим больше всего страдал на дипломе, потому что testvkr это как-то очень <<весело>> проверял...

\textbf{Э: }Ну он буквально это проверяет так, как он это видит посимвольно, он читает текст и сравнивает.

\textbf{И: }А нельзя ли использовать простой способ отличия? Условно... А, мы же можем ссылаться и внутри предложения, и за его пределами, после точки.

\textbf{Э: }Да.

\textbf{И: }Причём даже не упоминая его перед этим словами <<вот у нас в этом источнике...>>.

\textbf{Э: }У нас может быть ссылка к предложению, к абзацу, к структурному элементу таблицы, картинки так далее. Может быть ссылка к отдельному слову внутри предложения...

\textbf{И: }Имеет ли смысл пытаться отличить другой вариант. То есть, когда мы, условно, пишем какой-то отрезок так, чтобы мы это отличали, условно, паттернами рядом стоящих слов. Условно <<в отрезке таком-то искали>>. Вот такое словосочетание. Есть ли смысл пытаться такие паттерны отлавливать?

\textbf{Э: }Возможно имеет смысл. Вопрос: на каком этапе? Ну то есть, допустим, testvkr, так как он всё равно применяется в комплекте с нормоконтролером, а не по отдельности, то есть возможность, когда мы просто игнорируем часть замечаний, которые не являются замечаниями по существу, а действительно просто программная неточность. Ошибкой это тоже не назовёшь.

\textbf{И: }Ну про полную замену нормоконтролера мы уже говорили. Так, на этом у меня, кажется, вопросы закончились. Я вроде всё для себя установил. Спасибо большое, что уделили время.

%\section*{Вывод}
%
%<что сделали в конструкторке и получили в результате, кратко>

\clearpage
