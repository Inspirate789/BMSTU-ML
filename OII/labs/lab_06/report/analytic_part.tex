\chapter{Аналитическая часть}



\section{Постановка задачи}

Электронные отчёты по ЛР студенто представляют собой многокомпонентные документы, состоящие из множества элементов (титульный лист, содержание, основная часть, приложения и т.д.). Для успешной сдачи отчёта необходимо его строгое соответствие установленным требованиям по
\begin{enumerate}[label*=---]
	\item структуре документа (наличие разделов, их последовательность);
	\item форматированию (шрифты, размеры текста, отступы);
	\item содержанию (корректность данных, орфография);
	\item визуальной компоновке (поля, расположение заголовков, читабельность).
\end{enumerate}

Однако, студенты часто допускают различные ошибки: неверное оформление титульного листа, отсутствие обязательных разделов, неправильно оформленные таблицы, орфографические или структурные ошибки.

Для автоматизации проверки отчётов предлагается разработать программный комплекс для выявления ошибок в отчётах. Основное внимание уделяется обработке сканов или скриншотов электронных отчётов, что по сути является задачей анализа изображения, включающей в себя
\begin{enumerate}[label*=---]
	\item распознавание содержимого документа;
	\item выделение ключевых элементов (заголовки, текстовые блоки, таблицы);
	\item поиск ошибок, основанный как на структурных, так и на содержательных требованиях.
\end{enumerate}


\clearpage
