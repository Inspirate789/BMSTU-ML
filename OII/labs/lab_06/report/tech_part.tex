\chapter{Технологическая часть}

\section{Средства реализации}

Для автоматизации анализа электронных отчётов первым шагом является обработка самого изображения (фотографии или скана). Поскольку отчёты могут быть сканированы под углом, с тенями, шумами или недостаточной читаемостью текста, необходимо предусмотреть возможность
\begin{enumerate}[label*=---]
	\item устранения искажений изображений (перспективных, геометрических), удаления шумов, коррекции яркости и контраста;
	\item разделения документов на области;
	\item распознавания текста в документах;
	\item проверки соответствия визуальных элементов документов заданным шаблонам.
\end{enumerate}

Для выполнения описанных задач наиболее актуальным инструментом является библиотека OpenCV~\cite{opencv} для языка программирования Python~\cite{pythonlang}, которую предлагается использовать в данной работе.

\section{Реализация алгоритма}

На листинге \ref{lst:1} представлена реализация программы для определения нарушения интервала после и до заголовка раздела (подраздела).

\begin{lstlisting}[label=lst:1,caption= Сравнение алгоритмов кластеризации K-Means и K-Medoids]
import cv2

filename = "neg2.png"
relative_indent_threshold = 1.2
relative_text_height_threshold = 1.35

img = cv2.imread(filename)
gray = cv2.cvtColor(img, cv2.COLOR_BGR2GRAY)
ret, thresh1 = cv2.threshold(gray, 0, 255, cv2.THRESH_OTSU | cv2.THRESH_BINARY_INV)

# Specify structure shape and kernel size. 
# Kernel size increases or decreases the area of the rectangle to be detected.
# A smaller value like (10, 10) will detect each word instead of a sentence.
rect_kernel = cv2.getStructuringElement(cv2.MORPH_RECT, (1, 1))

dilation = cv2.dilate(thresh1, rect_kernel, iterations = 1)
contours, hierarchy = cv2.findContours(dilation, cv2.RETR_EXTERNAL, cv2.CHAIN_APPROX_NONE)
res = img.copy()
rects = [cv2.boundingRect(cnt) for cnt in contours]

class ContourComparator(tuple):
	def __lt__(self, other):
		return self[1] < other[1] or self[0] < other[0]

rects.sort(key=ContourComparator)

i = 0
while i < len(rects) - 1:
	(x, y, w, h) = rects[i]
	(x_next, y_next, w_next, h_next) = rects[i+1]
	if y_next <= y + h:
		x_new = min(x, x_next)
		y_new = min(y, y_next)
		w_new = max(x + w, x_next + w_next) - x_new
		h_new = max(y + h, y_next + h_next) - y_new
		rects[i] = (x_new, y_new, w_new, h_new)
		del rects[i+1]
	else:
		i += 1

rects = [(x, y, w, h) for (x, y, w, h) in rects if w / h > 1]
spaces = [(
	5, 
	rects[i][1] + rects[i][3], 
	img.shape[1] - 10, 
	rects[i+1][1] - rects[i][1] - rects[i][3],
) for i in range(len(rects) - 1)]

h_min = rects[0][3]
for rect in rects:
	print(rect)
	(x, y, w, h) = rect
	if h < h_min:
		h_min = h
	# cv2.rectangle(res, (x, y), (x + w, y + h), (0, 255, 0), 2)

print(f"min h = {h_min}\n")

image_correct = True
for i in range(len(spaces) - 1):
	print(spaces[i])
	(x, y, w, h) = spaces[i]
	(x_next, y_next, w_next, h_next) = spaces[i+1]
	text_height = rects[i+1][3]
	#cv2.rectangle(res, (x, y), (x + w, y + h), (255, 0, 0), 2)
	if text_height/h_min > relative_text_height_threshold and \
		max(h, h_next) / min(h, h_next) >= relative_indent_threshold:
		image_correct = False
		cv2.rectangle(res, (x, y), (x + w, y + h), (0, 0, 255), 2)
		cv2.rectangle(res, (x_next, y_next), (x_next + w_next, y_next + h_next), (0, 0, 255), 2)

if not image_correct:
	cv2.imwrite("recognized_" + filename, res)
	print(f"image has errors, see recognized_{filename}")
else:
	print("image has no errors")

\end{lstlisting}

\clearpage
