\ssr{ВВЕДЕНИЕ}

В современных социологических исследованиях часто возникает проблема неполноты данных, связанная как с отказом респондентов отвечать на отдельные вопросы, так и с техническими сложностями при сборе информации. Одновременно с этим, понимание взаимосвязей между различными социальными показателями представляет существенный интерес для исследователей. Методы машинного обучения предоставляют эффективный инструментарий для решения обеих задач: выявления корреляционных зависимостей между переменными и восстановления пропущенных значений на основе имеющихся данных.

Даны результаты опроса населения о его условиях существования. Переменные разбиты на 2 класса --- <<Признаки состояния>> --- это субъективная оценка населения своего бытия и <<Признаки причины>> --- объектные количественные признаки оценивающие жизнедеятельность индивида и социума, в котором он проживает.

К признакам состояния относятся:
\begin{enumerate}[label*=\arabic*.]
	\item Оценка благополучия.
	\item Оценка социальной поддержки.
	\item Ожидаемая продолжительность здоровой жизни.
	\item Свобода граждан самостоятельно принимать жизненно важные решения.
	\item Индекс Щедрости.
	\item Индекс отношения к коррупции.
	\item Оценка риска безработицы.
	\item Индекс кредитного оптимизма.
	\item Индекс страха социальных конфликтов.
	\item Индекс семьи.
	\item Индекс продовольственной безопасности.
	\item Чувство технологического прогресса.
	\item Чувство неравенства доходов в обществе.
\end{enumerate}

К индивидуальным признакам причины относятся:
\begin{enumerate}[label*=\arabic*.]
	\item Среднегодовой доход, тыс. \$.
	\item Объем потребленного алкоголя в год, л.
	\item Количество членов семьи.
	\item Количество лет образования.
	\item Доля от дохода семьи, которая тратится на продовольствие, \%.
\end{enumerate}

К общественным признакам причины относятся:
\begin{enumerate}[label*=\arabic*.]
	\item Коэффициент Джини сообщества --- показатель степени расслоения общества по какому-либо социальному признаку. Одними из ключевых признаков, по которым рассчитывается коэффициент Джини, является уровень доходов и активов домохозяйств. Показатель может варьироваться в диапазоне от 0 до 1, и чем больше его значение, тем большее расслоение общества он отражает.
	\item Издержки сообщества на окружающую среду, млн. \$.
	\item Охват беспроводной связи в сообществе, \%.
	\item Количество смертей от вирусных и респираторных заболеваний в сообществе, тыс. человек.
	\item Волатильность потребительских цен в сообществе.
	\item Индивидуальные показатели характеризуют непосредственно индивида, общественные - сообщество в котором он проживает. В выборке могут присутствовать по несколько человек из одного сообщества. Все их общественные характеристики таким образом будут совпадать. В данных, относящихся к признакам состояния, присутствуют пропуски.
\end{enumerate}

Целью данной лабораторной работы является освоение практических навыков корреляционного анализа.

Задачи данной лабораторной работы:
\begin{enumerate}[label*=\arabic*)]
	\item определить влияние признаков причины на признаки состояния;
	\item выявить корреляционные зависимости; 
	\item заполнить пропуски в данных.
\end{enumerate}

\clearpage
