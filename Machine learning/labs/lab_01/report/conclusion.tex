\ssr{ЗАКЛЮЧЕНИЕ}

В рамках лабораторной работы была изучена модель полиномиальной регрессии и феномен Рунге. Все поставленные задачи были выполнены.

\begin{enumerate}[label*=\arabic*)]
	\item Создана обучающая выборка с использованием функции $y(x)=\theta_1x+\theta_2sin(x)+\theta_3$ и с добавлением шума с нормальным распределением;
	\item Построена модель полиномиальной регрессии, аппроксимирующей данные обучающей выборки;
	\item Найдена оптимальная степень полинома для аппроксимации.
	\item Рассмотрен феномен (явление) Рунге;
	\item Рассчитан функционал эмпирического риска (функционал качества) для обучающей и контрольной выборок (выведены графики);
	\item Оценена обобщающая способность;
	\item Найдена оптимальная степень полинома для аппроксимации;
\end{enumerate}

Оптимальная степень полинома для аппроксимации зависимости $y(x)=\theta_1x+\theta_2sin(x)+\theta_3$  --- 6.

Оптимальная степень полинома для аппроксимации зависимости $y(x)=\frac{1}{1+25x^2}$  --- 20. Феномен Рунге можно наблюдать при степени полинома 55.
