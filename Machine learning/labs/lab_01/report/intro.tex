\ssr{ВВЕДЕНИЕ}

Полиномиальная регрессия представляет собой одну из наиболее широко используемых моделей машинного обучения, нацеленную на аппроксимацию нелинейной зависимости между параметрами. Эта модель служит основным инструментом для прогнозирования значений целевой функции на основе набора входных переменных.

Однако, несмотря на простоту и эффективность этой модели при определённых условиях, она не всегда обеспечивает точные результаты. Практика показывает, что если степень полинома слишком велика по отношению к размеру обучающей выборки или если данные содержат явные или неточные измерения, то модель может стать крайне чувствительной ко всем шумам в данных, что приводит к переобучению. Этот феномен известен как феномен Рунге и является классическим примером одной 
из наиболее распространённых проблем машинного обучения — проблемы переоценки (переобучения).

Переоценка происходит тогда, когда модель начинает слишком точно подгоняться к данным, а не нацеливается на выявление закономерностей в данных. Это может привести к тому, что модель будет прогнозировать значения целевой переменной с помощью случайных шумов, которые содержатся в исходном наборе данных.

Целью данной лабораторной работы изучение полиномиальной регрессии и феномена Рунге.

Задачи данной лабораторной работы:
\begin{enumerate}[label*=\arabic*)]
	\item создать обучающую выборку с использованием функции $y(x)=\theta_1x+\theta_2sin(x)+\theta_3$ и с добавлением шума с нормальным распределением;
	\item построить модель полиномиальной регрессии, аппроксимирующей данные обучающей выборки;
	\item найти оптимальную степень полинома для аппроксимации;
	\item рассмотреть феномен (явление) Рунге;
	\item рассчитать функционал эмпирического риска (функционал качества) для обучающей и контрольной выборок (вывести графики);
	\item оценить обобщающую способность;
	\item найти оптимальную степень полинома для аппроксимации.
\end{enumerate}

\clearpage
