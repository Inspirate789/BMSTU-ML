\chapter{Аналитическая часть}



\section{Модель полиномиальной регрессии}

Полиномиальная регрессия является одним из наиболее широко используемых алгоритмов машинного обучения, нацеленных на аппроксимацию нелинейной зависимости между переменными. Основная 
идея состоит в том, чтобы найти полиномиальное выражение для целевой переменной на основе набора входных переменных.

Пусть $\mathbf{x}$ является набором входных переменных и $y$ — целевой функцией, которую мы хотим предсказать. Цель полиномиальной регрессии — найти коэффициенты $w = (w_0, w_1, ..., 
w_n)$ в таких, что

$$y \approx w^T\phi(\mathbf{x})$$где $\phi$ — функция, которая преобразует исходные данные в набор признаков.



\section{Феномен Рунге}

Феномен Рунге является классическим примером проблемы переоценки в контексте полиномиальной регрессии. Этот феномен был впервые описан Карлом Давидом Рунге в 1880 году и относится 
к проблеме решения разностных уравнений.

В контексте полиномиальной регрессии феномен Рунге проявляется в следующем: когда степень полинома слишком велика по отношению к размеру обучающей выборки, модель начинает 
воспроизводить не закономерности в данных, а случайные шумы. Это приводит к тому, что точность её прогнозов снижается.

\clearpage
