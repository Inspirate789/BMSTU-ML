\ssr{ВВЕДЕНИЕ}

Кластеризация является одной из важнейших задач в области анализа данных. Она используется для группировки объектов в такие подмножества (кластеры), в которых объекты
внутри каждого кластера максимально схожи, а объекты из разных кластеров максимально различны. Задача кластеризации широко применяется в различных областях, включая обработку
текстов, анализ изображений, биоинформатику и многие другие.

Целью данной лабораторной работы является изучение алгоритмов кластеризации на примере кластериного анализа результатов социологического исследования. Для достижения поставленной цели необходимо выполнить следующие задачи.
\begin{enumerate}[label*=\arabic*.]
	\item Провести кластерный анализ данных. Использовать не менее двух алгоритмов кластеризации (например: иерархический, К-средних, DBSCAN и др.). Варьировать различные значения гиперпараметров и тип расстояний.
	\item  Оценить работу алгоритмов с использованием внешних и внутренних мер оценки качества (в том числе, построить таблицу сопряжённости с учётом классов Шкалы Кантрила). Определить оптимальное количество кластеров и их структуру.
\end{enumerate}

\clearpage
