\chapter{Конструкторская часть}

\section{Генетический алгоритм}

Генетический алгоритм начинается с популяции случайно выбранных потенциальных решений (индивидуумов), для которых вычисляется функция приспособленности. Алгоритм выполняет цикл, в котором последовательно применяются операторы отбора, скрещивания и мутации, после чего приспособленность индивидуумов пересчитывается. Цикл продолжается, пока не выполнено условие остановки, после чего лучший индивидуум в текущей популяции считается решением.

Начальная популяция состоит из случайным образом выбранных потенциальных решений (индивидуумов). Поскольку в генетических алгоритмах индивидуумы представлены хромосомами, начальная популяция ---  это, по сути дела, набор хромосом. Формат хромосом должен соответствовать принятым для решаемой задачи правилам, например это могут быть двоичные строки определённой длины. В случае аппроксимации функций хромосомой будет являться значение схожести аппроксимированного и истинного значения функции в точке. Данное значение предлагается рассчитывать по формулам

\label{fitness}
\begin{equation}
	fitness =  ARI = \frac{RI - E[RI]}{max(RI) - E[RI]},
\end{equation}

\begin{equation}
	RI = \frac{a + b}{{{n \choose 2}}},
\end{equation}

\begin{equation}
	E[RI] = \frac{\sum_i {{n_i \choose 2}} \sum_j {{m_j \choose 2}}}{{n \choose 2}},
\end{equation}
где $a$ --- количество пар элементов, которые находятся в одном кластере как в истинной, так и в прогнозируемой кластеризации, $b$ --- количество пар элементов, которые находятся в разных кластерах как в истинной, так и в прогнозируемой кластеризации, $n$ --- общее количество образцов, $n_i$ --- количество образцов в истинном кластере i, $m_j$ --- количество образцов в прогнозируемом кластере j.

Для каждого индивидуума вычисляется функция приспособленности. Это делается один раз для начальной популяции, а затем для каждого нового поколения после применения операторов отбора, скрещивания и мутации. Поскольку приспособленность любого индивидуума не зависит от всех остальных, эти вычисления можно производить параллельно.

Так как на этапе отбора, следующем за вычислением приспособленности, более приспособленные индивидуумы обычно считаются лучшими решениями, генетические алгоритмы естественно «заточены» под нахождение максимумов функции приспособленности. Если в какой-то задаче нужен минимум, то при вычислении приспособленности следует инвертировать найденное значение, например умножив его на –1.

Применение генетических операторов к популяции приводит к созданию новой популяции, основанной на лучших индивидуумах из текущей.

Оператор \textbf{отбора} отвечает за отбор индивидуумов из текущей популяции таким образом, что предпочтение отдаётся лучшим.

Оператор \textbf{скрещивания} (или рекомбинации) создаёт потомка выбранных индивидуумов. Обычно для этого берутся два индивидуума, и части их хромосом меняются местами, в результате чего создаются две новые хромосомы, представляющие двух потомков.

Оператор \textbf{мутации} вносит случайные изменения в один или несколько генов хромосомы вновь созданного индивидуума. Обычно вероятность мутации очень мала.

\clearpage
