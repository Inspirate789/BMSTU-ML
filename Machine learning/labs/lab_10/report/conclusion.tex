\ssr{ЗАКЛЮЧЕНИЕ}

В рамках лабораторной работы было проведено изучение эволюционных алгоритмов на примере решения задачи кластеризации.

\begin{enumerate}[label*=\arabic*.]
	\item Проведён кластерный анализ данных с использованием агломеративной кластеризации и алгоритма HDBSCAN с варьированием различных значений гиперпараметров и типов расстояний.
	\item  Проведена оценка работы алгоритмов с использованием внешних и внутренних мер оценки качества. Определено оптимальное количество кластеров и их структура.
\end{enumerate}

Максимальное значение метрики ARI (0.37) было достигнуто при использовании агломеративной кластеризации векторов, полученных после снижения размерности исходных данных с помощью алгоритма UMAP. В алгоритме использовалось манхеттенское расстояние и попарное среднее расстояние между кластерами. В результате было получено 4 кластера, что ближе к числу классов в укрупнённой шкале Кантрила (3 класса), чем к стандартной (10 классов).
