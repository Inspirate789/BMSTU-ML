\ssr{ВВЕДЕНИЕ}

В современной теории машинного обучения задача классификации является одной из фундаментальных проблем, имеющей широкий спектр практических приложений. Наивный байесовский классификатор представляет собой важный базовый алгоритм, демонстрирующий ключевые принципы вероятностного подхода к классификации данных.

Целью данной лабораторной работы является изучение принципов функционирования наивного байесовского классификатора.

Задачи данной лабораторной работы:
\begin{enumerate}[label*=\arabic*)]
	\item осуществить предобработку данных, сформировать обучающий и тестовый наборы данных;
	\item построить наивный байесовский классификатор;
	\item оценить качество работы классификатора.
\end{enumerate}

\clearpage
