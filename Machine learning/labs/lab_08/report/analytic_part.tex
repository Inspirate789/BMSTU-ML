\chapter{Аналитическая часть}

\section{Классификация}

Классификация (classification) — это задача присвоения меток класса
(class label) наблюдениям (Observation) объектам из предметной области.
Множество допустимых меток класса конечно. В свою очередь класс —
это множество всех объектов с данным значением метки. Требуется
построить алгоритм, способный классифицировать (присвоить метку)
произвольный объект из исходного множества. Классификация, как
правило, на этапе настройки использует обучение с учителем.

\section{Деревья решений}

Деревья решений относятся к методам поиска логических
закономерностей в данных, а также являются основным подходом,
применимым в теории принятия решений.

Они позволяют осуществлять решение целого класса задач
классификации и регрессии в виде многошагового процесса принятия
решений и используют особенности древовидных классификаторов,
связанных с учётом локальных свойств классифицируемых объектов на
каждом уровне и в каждом узле дерева, что позволяет реализовать как
прямую, так и обратную цепочку рассуждений.

Основными достоинствами деревьев
решений является

\begin{itemize}[label*=---]
	\item простота и наглядность описания
	процесса поиска решения;
	\item с точки зрения математики, обучение нейронных сетей --- это
	многопараметрическая задача нелинейной оптимизации.
	\item представление правил в виде
	продукций <<если… то…>>.
\end{itemize}

\section{Ансамбли}

Ансамбли --- это контролируемые алгоритмы обучения, которые комбинируют
прогнозы из двух и более алгоритмов машинного обучения для построения более
точных результатов. Результаты можно комбинировать с помощью голосования или
усреднения. Первое зачастую применяется в классификации, а второе --- в
регрессии.

Существует 3 основных типа ансамблевых алгоритмов.

Обучение персептрона:
\begin{enumerate}[label*=\arabic*.]
	\item \textbf{Бэггинг}. Алгоритмы обучаются и работают параллельно на разных
	тренировочных наборах одного размера. Затем все они тестируются на одном
	наборе данных, а конечный результат определяется с помощью голосования.
	\item \textbf{Бустинг}. В этом типе алгоритмы обучаются последовательно, а конечный
	результат отбирается с помощью голосования с весами.
	\item \textbf{Стекинг (наложение}). Исходя из названия, этот подход состоит из двух уровней,
	расположенных друг на друге. Базовый представляет собой комбинацию
	алгоритмов, а верхний --- мета-алгоритмы, основанные на базовом уровне.
\end{enumerate}



\clearpage
