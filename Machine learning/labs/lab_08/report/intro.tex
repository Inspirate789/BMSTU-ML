\ssr{ВВЕДЕНИЕ}

Классификация данных представляет собой одну из ключевых задач машинного обучения, для решения которой разработано множество эффективных алгоритмов. Среди них особое место занимают деревья решений и ансамблевые методы, демонстрирующие высокую интерпретируемость результатов и устойчивость к переобучению.

Целью данной лабораторной работы является изучение деревьев решений и ансамблевых классификаторов на примере анализа социологического исследования.

Задачи данной лабораторной работы:
\begin{enumerate}[label*=\arabic*)]
	\item определить, какие из признаков состояния наиболее сильно связаны с интегральной оценкой счастья (благополучия) респондента;
	\item пользуясь найденными закономерностями спрогнозировать попадание респондентов, у которых интегральная характеристика отмечена как "Неизвестно", в укрупнённые группы шкалы Кантрила;
	\item построить следующие классификаторы: многоклассовую логистическую регрессию, дерево решений, ансамблевый классификатор;
	\item сравнить матрицы ошибок и метрики качества классификации.
\end{enumerate}

\clearpage
