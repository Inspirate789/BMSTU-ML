\ssr{ЗАКЛЮЧЕНИЕ}

В рамках лабораторной работы было проведено изучение деревьев решений и ансамблевых классификаторов на примере анализа социологического исследования.

\begin{enumerate}[label*=\arabic*.]
	\item Определено, какие из признаков состояния наиболее сильно связаны с интегральной оценкой счастья (благополучия) респондента;
	\item Спрогнозировано попадание респондентов, у которых интегральная характеристика отмечена как <<Неизвестно>>, в укрупнённые группы шкалы Кантрила;
	\item Построены следующие классификаторы: многоклассовая логистическая регрессия, дерево решений, ансамблевый классификатор;
	\item Проведено сравнение матриц ошибок и метрик качества классификации.
\end{enumerate}

Наиболее точным классификатором оказалась ансамблевая модель AdaBoost над деревьями регрессии, для которой максимальное значение метрики MCC составила 0.952.
