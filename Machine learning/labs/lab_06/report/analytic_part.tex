\chapter{Аналитическая часть}

\section{Классификация}

Классификация (classification) — это задача присвоения меток класса
(class label) наблюдениям (Observation) объектам из предметной области.
Множество допустимых меток класса конечно. В свою очередь класс —
это множество всех объектов с данным значением метки. Требуется
построить алгоритм, способный классифицировать (присвоить метку)
произвольный объект из исходного множества. Классификация, как
правило, на этапе настройки использует обучение с учителем.

\section{Линейный дискриминант Фишера}

\textbf{Линейный дискриминантный анализ} (ЛДА), а также связанный с ним
линейный дискриминант Фишера --- методы статистики и машинного обучения,
применяемые для нахождения линейных комбинаций признаков, наилучшим
образом разделяющих два или более класса объектов или событий. Полученная
комбинация может быть использована в качестве линейного классификатора или
для сокращения размерности пространства признаков перед последующей
классификацией.

\textbf{Линейный дискриминант Фишера} в первоначальном значении - метод,
определяющий расстояние между распределениями двух разных классов объектов
или событий. Он может использоваться в задачах машинного обучения при
статистическом (байесовском) подходе к решению задач классификации.

Предположим, что обучающая выборка удовлетворяет помимо базовых
гипотез байесовского классификатора также следующим гипотезам:
\begin{itemize}[label*=---]
	\item классы распределены по нормальному закону;
	\item матрицы ковариаций классов равны.
\end{itemize}
Тогда статистический подход приводит к линейному дискриминанту, и именно
этот алгоритм классификации в настоящее время часто понимается под термином
\textit{линейный дискриминант Фишера}.



\clearpage
