\ssr{ВВЕДЕНИЕ}

В теории машинного обучения байесовский подход к классификации представляет собой фундаментальную методологию, основанную на строгом математическом аппарате теории вероятностей. Данный подход позволяет не только осуществлять классификацию объектов, но и получать вероятностные оценки принадлежности объекта к каждому из классов.

Целью данной лабораторной работы является изучение линейного дискриминанта Фишера на примере построения классификатора <<Ирисов Фишера>> с использованием байесовского подхода.

Задачи данной лабораторной работы:
\begin{enumerate}[label*=\arabic*)]
	\item осуществить исследование и подготовку исходных данных;
	\item построить гистограммы распределения значений для каждого признака и для каждого класса;
	\item произвести визуализацию проекций классов на плоскости, где по осям  отложены различные комбинации пар признаков;
	\item построить матрицы корреляций между различными признаками, как для всей выборки в целом, так и для отдельных классов;
	\item построить классификатор с использованием байесовского подхода;
	\item оценить точность, полноту, F-меру; построить матрицу ошибок.
\end{enumerate}

\clearpage
