\ssr{ЗАКЛЮЧЕНИЕ}

В рамках лабораторной работы было проведено изучение нейросетевого подхода на примере построения классификатора на базе многослойного персептрона.

\begin{enumerate}[label*=\arabic*.]
	\item Осуществлена генерация исходных данных.
	\item Для сгенерированного датасета построен классификатор на базе многослойного персептрона.
	\item Дано обоснование выбора числа слоев и нейронов в каждом слое.
	\item Проведено сравнение работы нейросети в зависимости от выбранной функции активации (сигмоида с разными значениями параметра крутизны (b=1,b=100);  ReLU ).
	\item Обоснован момент остановки процесса обучения.
	\item Оценены точность, полнота, F-мера.
	\item Построена матрицу ошибок.
	\item Предусмотрена дополнительную возможность ввода пользователем новых, не входящих в сгенерированный датасет данных.
\end{enumerate}

Для построенного классификатора максимальное значение метрики MCC составила 0.9829.
