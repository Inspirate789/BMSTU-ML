\chapter{Конструкторская часть}


\section{Обучение моделей}

Для обучения моделей использовалась следующая архитектура нейронной сети:

\begin{itemize}[label*=---]
	\item Входной слой --- 2 нейрона (в соответствии с размерностью исходных данных).
	\item Скрытый слой --- 6 нейронов.
	\item Выходной слой --- 4 нейрона (в соответствии с количеством целевых классов).
\end{itemize}

В качестве функций активации по заданию лабораторной работы будут использоваться функции ReLu и логистическая с параметрами 1 и 5.

В качестве метрик обучения предлагается использовать accuracy, precision и recall.

\clearpage
