\ssr{ВВЕДЕНИЕ}

Искусственные нейронные сети, в частности многослойные персептроны, являются одним из ключевых инструментов в машинном обучении для решения задач классификации. Эти модели, основанные на принципе имитации работы биологических нейронных сетей, позволяют эффективно аппроксимировать сложные нелинейные зависимости и обрабатывать многомерные данные, что делает их универсальными классификаторами.

Целью данной лабораторной работы является изучение нейросетевого подхода на примере построения классификатора на базе многослойного персептрона.

Задачи данной лабораторной работы:
\begin{enumerate}[label*=\arabic*)]
	\item осуществить генерацию исходных данных, которые представляют собой двумерное признаковое пространство, сгруппированное в 6 или более областей, отнесённых не менее чем к 4 классам;
	\item для сгенерированного датасета осуществить построение классификатора на базе многослойного персептрона;
	\item обосновать выбор числа слоев и нейронов в каждом слое;
	\item сравнить работу нейросети в зависимости от выбранной функции активации (сигмоида с разными значениями параметра крутизны (b=1,b=100);  ReLU );
	\item обосновать момент остановки процесса обучения;
	\item оценить точность, полноту, F-меру;
	\item построить матрицу ошибок;
	\item предусмотреть дополнительную возможность ввода пользователем новых, не входящих в сгенерированный датасет данных.
\end{enumerate}

\clearpage
