\chapter{Аналитическая часть}

\section{Классификация}

Классификация (classification) — это задача присвоения меток класса
(class label) наблюдениям (Observation) объектам из предметной области.
Множество допустимых меток класса конечно. В свою очередь класс —
это множество всех объектов с данным значением метки. Требуется
построить алгоритм, способный классифицировать (присвоить метку)
произвольный объект из исходного множества. Классификация, как
правило, на этапе настройки использует обучение с учителем.

\section{Искусственная нейронная сеть}

Математическая модель, а также ее программные или аппаратные
реализации, построенная в некотором смысле по образу и подобию сетей
нервных клеток живого организма.
Под искусственной нейронной сетью в дальнейшем будем понимать сеть
искусственных нейронов, соединенных между собой. Здесь предполагается, что
нейроны могут соединятьcя между собой произвольным образом и
образовывать таким образом разнообразные нейронные структуры.

\begin{itemize}[label*=---]
	\item С точки зрения машинного обучения, нейронная сеть представляет собой
	частный случай методов распознавания образов, дискриминантного
	анализа.
	\item С точки зрения математики, обучение нейронных сетей --- это
	многопараметрическая задача нелинейной оптимизации.
	\item С точки зрения кибернетики, нейронная сеть используется в задачах
	адаптивного управления и как алгоритмы для робототехники.
	\item С точки зрения развития вычислительной техники и программирования,
	нейронная сеть --- способ решения проблемы эффективного параллелизма.
	\item С точки зрения искусственного интеллекта, ИНС является основой
	философского течения коннекционизма и основным направлением в
	структурном подходе по изучению возможности построения
	(моделирования) естественного интеллекта с помощью компьютерных
	алгоритмов.
\end{itemize}

\section{Методы обучения многослойного персептрона}

Обучение нейронной сети --- интерактивный процесс корректировки
синаптических весов и порогов. В процессе обучения нейронная сеть получает
и обобщает знания об окружающей среде и тех данных, с которыми ей
придется оперировать. Существуют два концептуальных подхода к обучению
нейронных сетей: обучение с учителем и обучение без учителя.

Обучение персептрона:
\begin{itemize}[label*=---]
	\item обучение Хебба (без учителя);
	\item стохастические методы обучения (<<имитация обжига>>);
	\item обратное распространение ошибки (Backpropagation).
\end{itemize}



\clearpage
