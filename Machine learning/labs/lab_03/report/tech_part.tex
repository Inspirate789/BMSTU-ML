\chapter{Технологическая часть}

\section{Средства реализации}

В качестве языка программирования для реализации алгоритмов был выбран язык программирования Python ввиду наличия библиотек для обучения регрессионных моделей, таких как sklearn и numpy.

\section{Реализация алгоритмов}

На листинге \ref{lst:1} представлена реализация алгоритма проверки статистических гипотез для двух выборок.

\begin{lstlisting}[label=lst:1,caption=Проверка статистических гипотез для двух выборок]
	import numpy as np
	import scipy.stats as stats
	import matplotlib.pyplot as plt
	import seaborn as sns
	import math
	
	a = 0
	sigma = 1
	n = m = 30
	delta = 0.1
	alpha = 0.05
	
	x = np.random.normal(a, sigma, n)
	y = np.random.normal(a, sigma, n)
	x_0, y_0 = x, y
	
	plt.hist(x, alpha=0.5, label='x')
	plt.hist(y, alpha=0.5, label='y')
	plt.legend(loc='upper right')
	plt.show()
	
	shapiro_test1 = stats.shapiro(x)
	shapiro_test2 = stats.shapiro(y)
	
	print("@Тест Шапиро-Уилка для выборки@ 1: Statistic =", shapiro_test1.statistic, "p-value =", shapiro_test1.pvalue)
	print("@Тест Шапиро-Уилка для выборки@ 2: Statistic =", shapiro_test2.statistic, "p-value =", shapiro_test2.pvalue)
	
	t_stat, p_value = stats.ttest_ind(x, y, equal_var=True)
	print("@Тест Стьюдента для выборок@: Statistic =", t_stat, "p-value =", p_value)
	
	a_new = a
	t_stats, p_values = [], []
	rejected = False
	
	while not rejected:
		t_stat, p_value = stats.ttest_ind(x, y, equal_var=True)
		rejected = p_value < alpha
		t_stats.append(t_stat)
		p_values.append(p_value)
		if not rejected:
			a_new += delta
			y += delta
	
	print('@Итоговый сдвиг второй выборки@:', a_new - a)
	x_4, y_4 = x, y
	
	plt.plot(t_stats, label="t-statistic")
	plt.plot(p_values, label="P-value")
	plt.axhline(alpha, color="red", linestyle="--")
	plt.legend()
	plt.show()
	
	plt.hist(x_4, alpha=0.5, label='x')
	plt.hist(y_4, alpha=0.5, label='y')
	plt.legend(loc='upper right')
	plt.show()
	
	a_new = (a + a_new) / 2
	y -= (a_new - a) / 2
	t_stats, p_values = [], []
	rejected = False
	
	while not rejected:
		t_stat, p_value = stats.ttest_ind(x, y, equal_var=True)
		rejected = p_value < alpha
		t_stats.append(t_stat)
		p_values.append(p_value)
		if not rejected:
			x = np.hstack((x, np.random.normal(a, sigma, int(n*delta))))
			y = np.hstack((y, np.random.normal(a_new, sigma, int(n*delta))))
	
	print('@Размеры выборок@: len(x)=', len(x), 'len(y)=', len(y))
	x_5, y_5 = x, y
	
	plt.plot(t_stats, label="t-statistic")
	plt.plot(p_values, label="P-value")
	plt.axhline(alpha, color="red", linestyle="--")
	plt.legend()
	plt.show()
	
	plt.hist(x_5, alpha=0.5, label='x')
	plt.hist(y_5, alpha=0.5, label='y')
	plt.legend(loc='upper right')
	plt.show()
	
	conf_int_x = stats.t.interval(0.95, len(x_4)-1, loc=np.mean(x_4), scale=stats.sem(x_4))
	conf_int_y = stats.t.interval(0.95, len(y_4)-1, loc=np.mean(y_4), scale=stats.sem(y_4))
	
	print(f"95% @доверительный интервал для@ x: {conf_int_x}")
	print(f"@Ширина@ {conf_int_x[1] - conf_int_x[0]}")
	print(f"95% @доверительный интервал для@ y: {conf_int_y}")
	
	conf_int_x = stats.t.interval(0.95, len(x_5)-1, loc=np.mean(x_5), scale=stats.sem(x_5))
	conf_int_y = stats.t.interval(0.95, len(y_5)-1, loc=np.mean(y_5), scale=stats.sem(y_5))
	
	print(f"95% @доверительный интервал для@ x: {conf_int_x}")
	print(f"@Ширина@ {conf_int_x[1] - conf_int_x[0]}")
	print(f"95% @доверительный интервал для@ y: {conf_int_y}")
	
	t_dist = np.linspace(stats.t.ppf(0.001, n+m-2), stats.t.ppf(0.999, n+m-2), 1000)
	pdf_values = stats.t.pdf(t_dist, n+m-2)
	
	plt.plot(t_dist, pdf_values, label='@t-распределение@')
	plt.axvline(t_stat, color='r', linestyle='--', label='@Критическое значение@')
	plt.fill_between(t_dist, pdf_values, where=((t_dist < stats.t.ppf(alpha/2, n+m-2)) | (t_dist > stats.t.ppf(1 - alpha/2, n+m-2))), color='gray', alpha=0.5, label='@Область отклонения гипотезы@')
	plt.legend()
	plt.show()
	
	print(f'@Критическое значение@: {t_stat:.4f}')
	
	ci_low_a1, ci_high_a1 = stats.norm.interval(0.95, loc=np.mean(x_0), scale=stats.sem(x))
	ci_low_a2, ci_high_a2 = stats.norm.interval(0.95, loc=np.mean(y_0), scale=stats.sem(y))
	
	print(f'@Доверительный интервал для первой выборки@: [{ci_low_a1:.4f}, {ci_high_a1:.4f}]')
	print(f'@Доверительный интервал для второй выборки@: [{ci_low_a2:.4f}, {ci_high_a2:.4f}]')
\end{lstlisting}

\clearpage
