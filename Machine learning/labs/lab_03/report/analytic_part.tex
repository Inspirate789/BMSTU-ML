\chapter{Аналитическая часть}



\section{Критерий Шапиро-Уилка}
Критерий Шапиро-Уилка используется для проверки гипотезы $H_0$: «случайная величина X распределена нормально» и является одним наиболее эффективных критериев проверки нормальности. Критерии, проверяющие нормальность выборки, являются частным случаем критериев согласия.

Критерий Шапиро-Уилка основан на оптимальной линейной несмещённой оценке дисперсии к её обычной оценке методом максимального правдоподобия. Статистика критерия имеет вид:

\begin{equation}
	W=\frac{1}{s^2}\left[\sum_{i=1}^n a_{n-i+1} (x_{n-i+1} -x_i)\right]^2,
\end{equation}
где $s^2=\sum_{i=1}^n (x_i -\overline{x})^2, \overline{x}=\frac{1}{n}\sum_{i=1}^n x_i$.

Числитель является квадратом оценки среднеквадратического отклонения Ллойда. Коэффициенты $a_{n-i+1}$ берутся из таблиц. Критические значения статистики $W(\alpha)$ также находятся таблично.

Если $W<W(\alpha)$, то нулевая гипотеза о нормальности распределения отклоняется при уровне значимости $\alpha$. Приближённая вероятность получения эмпирического значения $W$ при $H_0$ вычисляется по формуле

\begin{equation}
	z=\gamma+\eta \ln \left(\frac{W-\epsilon}{1-W}\right),
\end{equation}
где $\gamma,\; \eta,\; \epsilon$ — табличные коэффициенты.

Критерий Шапиро-Уилка является очень мощным критерием для проверки нормальности, но, к сожалению, имеет ограниченную применимость. При больших значениях $n (n>100)$ таблицы коэффициентов $a_{n-i+1}$ становятся неудобными.


\section{Критерий Стьюдента}

Рассмотрим специальный случай двухвыборочных критериев согласия. Проверяется
гипотеза сдвига, согласно которой распределения двух выборок имеют
одинаковую форму и отличаются только сдвигом на константу.

\textbf{Критерий Стьюдента}. Рассмотрим теперь задачу сравнения средних
значений двух нормальных выборок.

Пусть $x_1,…,x_n$ и $y_1,…,y_m$ — нормальные независимые выборки из законов
распределения с параметрами $(a_1,\sigma_1^2 )$ и $(a_2,\sigma_2^2 )$ соответственно.

Рассмотрим проверку гипотезы:
\begin{equation}
	H_0: a_1 = a_2 против альтернативы H_1: a_1 \neq a_2
\end{equation}

Относительно параметров $\sigma_1^2$ и $\sigma_2^2$ выделим следующие четыре варианта
предположений:

\begin{enumerate}[label*=\arabic*)]
	\item обе дисперсии известны и равны между собой;
	\item обе дисперсии известны, но не равны между собой;
	\item обе дисперсии неизвестны, но предполагается, что они равны между собой;
	\item обе дисперсии неизвестны, их равенство не предполагается.
\end{enumerate}

Для построения критерия проверки гипотезы $H_0$ проведем
следующие рассуждения.

От выборок $x_1,…,x_n$ и $y_1,…,y_m$ перейдем к выборочным средним $\overline{x}$ и $\overline{y}$. Согласно свойствам нормального распределения и выдвинутой гипотезе, величины $\overline{x}$ и $\overline{y}$
имеют нормальные распределения с одними
тем же средним и дисперсиями $\sigma_1^2/n$ и $\sigma_2^2/m$.

Далее перейдем к статистике, основанной на выборочных средних $\overline{x}$ и $\overline{y}$ и дисперсиях $\sigma_1^2$ и $\sigma_2^2$ (если они известны) или их оценках $s_1^2$ и $s2^2$ (если дисперсии неизвестны).

	$\frac{1}{m}\sum\nolimits_{i=1}^m x_i$ — выборочное среднее,

	$\frac{1}{m-1}\sum\nolimits_{i=1}^m (x_i-\overline{x})^2$ — выборочная дисперсия.
	
Далее рассмотрим случай, когда обе дисперсии известны и равны между собой.

\begin{equation}
	\frac{\overline{x} - \overline{y}}{\sigma \sqrt{\frac{1}{n} + \frac{1}{m}}}
\end{equation}

Статистика имеет стандартное нормальное распределение, так как является
линейной комбинацией независимых нормальных величин. Гипотеза H
принимается на уровне значимости $\alpha$, если
\begin{equation}
	\left| \frac{\overline{x} - \overline{y}}{\sigma \sqrt{\frac{1}{n} + \frac{1}{m}}} \right| < z_{1-\alpha/2}
\end{equation}
в противном случае гипотеза отвергается в пользу альтернативы $a_1 \neq a_2$.



\clearpage
