\ssr{ЗАКЛЮЧЕНИЕ}

В рамках лабораторной работы была изучена модель полиномиальной регрессии и регуляризация. Все поставленные задачи были выполнены.

\begin{enumerate}[label*=\arabic*.]
	\item Сгенерировать две независимые выборки $x_1,…,x_n$ и $y_1,…,y_m$ с нормальные законом распределения и с параметрами  $(a_1,\sigma_1^2 )$ и $(a_2,\sigma_2^2 )$  соответствено;
	\item Осуществлена проверка гипотезы $H_0$ о соответствии выборок нормальному закону распределения; 
	\item Осуществлена проверка гипотезы $H_0$ о принадлежности выборок одной генеральной совокупности;
	\item Осуществлена проверка гипотезы $H_0: a_1 = a_2$ против альтернативы $H_1: a_1 \neq a_2$;
	\item Произведён сдвиг вправо всех элементов второй выборки на величину $\delta=0.01$ и осуществлена проверку гипотезы $H_0: a_1= a_2$ до тех пор, пока гипотеза $H_0$ не будет отвергнута;
	\item Для второй выборки назначено $a_2$ равным середине пройденного отрезка из пункта 5 и постепенно увеличивалось число элементов в выборках и осуществлялась проверку гипотезы $H_0: a_1= a_2$ до тех пор, пока гипотеза $H_0$ не будет отвергнута
	\item Рассчитаны 95\% доверительные интервалы для математических ожиданий двух выборок в момент, когда гипотеза $H_0$ была отвергнута в пунктах 5 и 6.
\end{enumerate}
