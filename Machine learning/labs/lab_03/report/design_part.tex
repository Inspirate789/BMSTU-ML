\chapter{Конструкторская часть}


\section{Обучение моделей}

Для обучения моделей полиномиальной регрессии мы будем использовать метод наименьших квадратов ( Least Squares Method ). Этот метод основан на минимизации функции потерь, представляющей сумму квадратичных отклонений данных от полученных предсказаний. Мы воспользуемся следующей функцией потерь:

\begin{equation}
L = \sum_{i}(y_i - f(x_i))^2,
\end{equation}
где  $y_i$ — фактические значения, а $f(x_i)$ — предсказанные значения для $i$-го наблюдения.


\section{Подбор степени полинома}

Следующий этап — подбор оптимальной степени полинома для модели. Для этого мы будем использовать информационный критерий Акаике (AIC). Критерий AIC определяется следующим образом:

\begin{equation}
AIC = 2k - 2log(L),
\end{equation}
где $k$ — количество параметров в модели, а $L$ — функция потерь. Таким образом, критерий не только вознаграждает за качество приближения, но и штрафует за использование излишнего количества параметров модели. Считается, что наилучшей будет модель с наименьшим значением критерия AIC.

На основе значения критерия AIC мы сможем выбрать модель с оптимальной степенью полинома, которая обеспечивает наилучшее соотношение точности и простоты модели.

\clearpage
