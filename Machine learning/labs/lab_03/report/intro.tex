\ssr{ВВЕДЕНИЕ}

Математическая статистика является фундаментальным компонентом машинного 
обучения, поскольку она обеспечивает необходимую базу для оценки 
достоверности выводов и принятия обоснованных решений на основе данных. В 
частности, проверка гипотез о математическом ожидании двух выборок имеет 
важное значение в статистике, поскольку позволяет исследователям оценить 
вероятность того, что две группы данных имеют схожие характеристики.

Целью данной лабораторной работы является освоение практических навыков 
проверки гипотез о математическом ожидании для двух случайных выборок с 
помощью статистических методов.

Задачи данной лабораторной работы:
\begin{enumerate}[label*=\arabic*)]
	\item сгенерировать две независимые выборки $x_1,…,x_n$ и $y_1,…,y_m$ с нормальные законом распределения и с параметрами  $(a_1,\sigma_1^2 )$ и $(a_2,\sigma_2^2 )$  соответствено;
	\item осуществить проверку гипотезы $H_0$ о соответствии выборок нормальному закону распределения; 
	\item осуществить проверку гипотезы $H_0$ о принадлежности выборок одной генеральной совокупности;
	\item осуществить проверку гипотезы $H_0: a_1 = a_2$ против альтернативы $H_1: a_1 \neq a_2$;
	\item производить сдвиг вправо всех элементов второй выборки на величину $\delta=0.01$ и осуществлять проверку гипотезы $H_0: a_1= a_2$ до тех пор, пока гипотеза $H_0$ не будет отвергнута;
	\item для второй выборки назначить $a_2$ равным середине пройденного отрезка из пункта 5 и постепенно увеличивать число элементов в выборках и осуществлять проверку гипотезы $H_0: a_1= a_2$ до тех пор, пока гипотеза $H_0$ не будет отвергнута
	\item рассчитать 95\% доверительные интервалы для математических ожиданий двух выборок в момент, когда гипотеза $H_0$ была отвергнута в пунктах 5 и 6.
\end{enumerate}

\clearpage
