\ssr{ВВЕДЕНИЕ}

В современной науке и инженерии методы машинного обучения играют всё более важную роль в решении сложных задач, таких как прогнозирование, классификация и анализ данных. Однако при 
работе с данными часто возникает проблема переобучения (overfitting), когда модель слишком хорошо подгоняется к конкретной выборке данных и перестаёт работать эффективно на новых, 
не встречавшихся ранее наблюдениях.

Одним из основных методов борьбы с проблемой переобучения является регуляризация. В рамках этой лабораторной работы мы рассмотрим основы полиномиальной регрессии и различные виды 
регуляризаций, используемых для предотвращения переобучения: L1-регуляризации (Lasso) и L2-регуляризации (Ridge).

Целью данной лабораторной работы изучение полиномиальной регрессии и регуляризации.

Задачи данной лабораторной работы:
\begin{enumerate}[label*=\arabic*)]
	\item для модели полиномиальной регрессии, полученной в ЛР 1 (п. 1) (оптимальный вариант), а также для полиномов больших  степеней (+5, +10) вывести значения коэффициентов полинома;
	\item к выбранным моделям (полиномам соответствующих степеней) применить метод регуляризации с использованием гребневой регрессии (ридж-регрессии) и Лассо-регрессии. 
	\item вывести значения коэффициентов полинома для различных значений параметра $\lambda$. 
	\item рассчитать функционал эмпирического риска (функционал качества) для всех полученных моделей на обучающей и контрольной выборках (вывести графики).
\end{enumerate}

\clearpage
