\ssr{ЗАКЛЮЧЕНИЕ}

В рамках лабораторной работы была изучена модель полиномиальной регрессии и регуляризация. Все поставленные задачи были выполнены.

\begin{enumerate}[label*=\arabic*)]
	\item Для модели полиномиальной регрессии, полученной в ЛР 1 (п. 1) (оптимальный вариант), а также для полиномов больших  степеней (+5, +10) выведены значения коэффициентов полинома;
	\item К выбранным моделям (полиномам соответствующих степеней) применён метод регуляризации с использованием гребневой регрессии (ридж-регрессии) и Лассо-регрессии. 
	\item Выведены значения коэффициентов полинома для различных значений параметра $\lambda$. 
	\item Рассчитан функционал эмпирического риска (функционал качества) для всех полученных моделей на обучающей и контрольной выборках (выведены графики).
\end{enumerate}

Ридж-регрессия позволяет добиться наибольшей точности обучения при степени полинома 11 и параметре регуляризации $\alpha = 1e-3$. Лассо-регрессия позволяет добиться наибольшей точности обучения при степени полинома 25 и $\alpha = 1e-3$.
