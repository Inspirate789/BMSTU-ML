\part*{ВВЕДЕНИЕ}
\addcontentsline{toc}{part}{\textbf{ВВЕДЕНИЕ}}

В телекоммуникациях индикатор силы принимаемого сигнала (Received Signal Strength Indicator, RSSI) --- это уровень мощности принимаемого радиосигнала, измеряемый по логарифмической шкале в дБм (dBm, децибел относительно 1 милливатта). Чем выше данное значение, тем качественнее сигнал, принимаемый от некоторого узла сети. Для устройств, работающих по стандартам Wi-Fi и Bluetooth 4.0~\cite{net}, RSSI является единственным параметром, позволяющим определить расстояние от устройства до базовой станции или маяка. Для вычисления расстояния используется модель потерь мощности сигнала на пути от источника до приёмника (Path loss model)~\cite{plm}, описываемая уравнением вида
\begin{equation}
	P_d = P_0 - 10 \cdot n \cdot lg⁡(\frac{d}{d_0}),
\end{equation}
где
\begin{itemize}[label*=---]
	\item $d$ --- искомое расстояние между источником и приёмником сигнала, м;
	\item $d_0$ --- опорное (калибровочное) расстояние, м;
	\item $P_d$ --- измеренная мощность сигнала (RSSI), дБм;
	\item $P_0$ --- мощность сигнала (RSSI), измеренная на расстоянии $d_0$, дБм;
	\item $n$ --- коэффициент потери мощности сигнала при распространении в среде, безразмерная величина, зависящая от свойств окружающей среды (например, наличия препятствий на пути прохождения сигнала).
\end{itemize}

Таким образом, для определения расстояния между двумя узлами беспроводной сети необходимо заранее знать коэффициент потери мощности сигнала $n$, а также путём калибровки оборудования определить $P_0$, выбрав произвольное расстояние $d_0$. Параметр $n$ можно оценить методом линейной регрессии. Это позволит получить модель, с помощью которой можно определять расстояние между источником и приёмником по мощности сигнала между ними. 

Для оценки прогнозов модели потерь мощности сигнала на пути от источника до приёмника необходима другая модель, учитывающая закон распределения RSSI. Такая модель может найти применение в построении ячеистых сетей (Mesh networks), в частности при разработке и тестировании алгоритмов функционирования мультиагентных систем  Для протокола BLE было показано, что величина RSSI распределена по нормальному закону\cite{my}. Примером модели, учитывающей нормальный закон распределения RSSI, является модель гауссовой смеси (Gaussian Mixture Model, GMM).

Целью данной работы является разработка вероятностной модели для оценки плотности распределения расстояния между узлами беспроводной сети в зависимости от уровня сигнала между ними. Для достижения поставленной цели необходимо решить следующие задачи.

\begin{enumerate}[label*=\arabic*.]
	\item Провести анализ предметной области и выбрать модель.
	\item Описать алгоритм построения выбранной модели.
	\item Реализовать программное обеспечение, которое будет моделировать зависимость расстояния между узлами беспроводной сети от уровня сигнала между ними. 
	\item Провести оценку плотности распределения расстояния между узлами беспроводной сети в зависимости от уровня сигнала между ними для протокола BLE.
\end{enumerate}